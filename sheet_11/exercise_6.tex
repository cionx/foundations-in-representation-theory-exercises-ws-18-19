\section{}





\subsection{}

I didn’t do this exercise, as it is nontrivial and I rather watched Trump’s attempt at a record length government shutdown.




\subsection{}

We have the following commutative diagram with (short) exact columns, in which the first two rows are (long) exact:
\[
  \begin{tikzcd}
      {}
    & 0
      \arrow{d}
    & 0
      \arrow{d}
    & 0
      \arrow{d}
    & 0
      \arrow{d}
    & {}
    \\
      \dotsb
      \arrow{r}
    & 0
      \arrow{r}
      \arrow{d}
    & X
      \arrow{r}[above]{a^0}
      \arrow{d}[left]{\id_X}
    & A^0
      \arrow{r}
      \arrow{d}
    & A^1
      \arrow{r}
      \arrow{d}
    & \dotsb
    \\
      \dotsb
      \arrow{r}
    & 0
      \arrow{r}
      \arrow{d}
    & X
      \arrow{r}[above]{i^0}
      \arrow{d}
    & I^0
      \arrow{r}
      \arrow{d}
    & I^1
      \arrow{r}
      \arrow{d}
    & \dotsb
    \\
      \dotsb
      \arrow{r}
    & 0
      \arrow{r}
      \arrow{d}
    & 0
      \arrow{r}
      \arrow{d}
    & B^0
      \arrow{r}
      \arrow{d}
    & B^1
      \arrow{r}
      \arrow{d}
    & \dotsb
    \\
      {}
    & 0
    & 0
    & 0
    & 0
    & {}
  \end{tikzcd}
\]
We regard the rows of this diagram as chain complexes and the overall diagram as a short exact sequence of these chain complexes.
The upper two rows are acyclic, hence the third row is acyclic by part~(i) of Exercise~2 of Exercise~sheet~8.





\subsection{}

We have for every~$n \geq 0$ a short exact sequence
\[
  0
  \to
  A^n
  \to
  I^n
  \to
  B^n
  \to
  0 \,.
\]
The object~$A^n$ is~\dash{$F$}{acyclic} and the object~$I^n$ is injective, and hence also~\dash{$F$}{acyclic}.
It follows from part~(i) of Exercise~5 that~$B^n$ is also~\dash{$F$}{acyclic}.





\subsection{}

We have for every~$n \geq 0$ the short exact sequence
\[
  0
  \to
  A^n
  \to
  I^n
  \to
  B^n
  \to
  0
\]
of~\dash{$F$}{acyclic} objects.
It follows from part~(i) of Exercise~5 that the sequence
\[
  0
  \to
  F(A^n)
  \to
  F(I^n)
  \to
  F(B^n)
  \to
  0
\]
is again (short) exact for every~$n \geq 0$.
This shows the exactness of the sequence
\[
  0 
  \to
  F(\Accc)
  \to
  F(\Iccc)
  \to
  F(\Bccc)
  \to
  0 \,,
\]
because this exactness is computed degreewise.





\subsection{}

We get from the short exact sequence~$0 \to F(\Accc) \to F(\Iccc) \to F(\Bccc) \to 0$ the long exact cohomology sequence
\[
  \dotsb
  \to
  \Hl^{n-1}(F(\Bccc))
  \to
  \Hl^n(F(\Accc))
  \to
  \Hl^n(F(\Iccc))
  \to
  \Hl^n(F(\Bccc))
  \to
  \dotsb
\]
The chain complex~$\Bccc$ is exact and consists of~\dash{$F$}{acyclic} objects, hence the chain complex~$F(\Bccc)$ is again exact by part~(iii) of Exercise~5.
We therefore have~$\Hl^n(F(\Bccc)) = 0$ for every~$n$.
We hence have for every~$n \geq 0$ an exact sequence
\[
  0
  \to
  \Hl^n(F(\Accc))
  \to
  \Hl^n(F(\Iccc))
  \to
  0 \,,
\]
which tells us that the morphism~$\Hl^n(F(\Accc)) \to \Hl^n(F(\Iccc))$ is an isomorphism.
We have that~$\Hl^n(F(\Iccc)) \cong (\Right^n F)(X)$ for every~$n \geq 0$ by the explicit computation of~$(\Right^n F)(X)$ via the injective resolution~$(\Iccc, i^0)$ of~$X$.









