\section{}





\subsection{}

The short exact sequence~$0 \to A' \to A \to A'' \to 0$ results in a long exact sequence
\[
  0
  \to
  F(A')
  \to
  F(A)
  \to
  F(A'')
  \to
  (\Right^1 F)(A')
  \to
  (\Right^1 F)(A)
  \to
  \dotsb
\]
We have for every~$n \geq 1$ that~$(\Right^n F)(A') = 0$ and~$(\Right^n F)(A) = 0$ because the objects~$A'$ and~$A$ are~\dash{$F$}{acyclic}.
We find for every~$n \geq 1$ from the exactness of the sequence
\[
  0
  \to
  \Right^n(A'')
  \to
  0
\]
that~$\Right^n(A'') = 0$, which shows that~$A''$ is also~\dash{$F$}{acyclic}.
We also find that the sequence
\[
  0
  \to
  F(A')
  \to
  F(A)
  \to
  F(A'')
  \to
  0
\]
is again exact.





\subsection{}

We set~$A^n \defined 0$ for every~$n < 0$.
It follows from the exactness of the sequence
\[
  \dotsb
  \to
  A^{n-1}
  \to
  A^n
  \to
  A^{n+1}
  \to
  \dotsb
\]
that we have for every~$n \in \Integer$ a short exact sequence
\[
  0
  \to
  Z^n
  \to
  A^n
  \to
  Z^{n+1}
  \to
  0 \,.
\]
We have for every~$n < 0$ that~$Z^n = 0$, and hence that~$Z^n$ is~\dash{$F$}{acyclic}.
If~$Z^n$ is acyclic then we get from the above short exact sequence and part~(i) of this exercise that~$Z^{n+1}$ is also~\dash{$F$}{acyclic}, since then both~$Z^n$ and~$A^n$ are~\dash{$F$}{acyclic}.
Whence we find inductively that~$Z^n$ is~\dash{$F$}{acyclic} for every~$n \in \Integer$.





\subsection{}

Instead of doing this exercise I spent my holidays being sick and watching UK policits going crazy.




