\section{}

We assume that the given quiver~$Q$ has only finitely many vertices and arrows.
We abbreviate~$A \defined kQ$.
Instead of representations of~$Q$ over~$k$ we will work with~{\modules{$A$}}.
Then~$X_i = \varepsilon_i X$ for every~{\module{$A$}}~$X$.





\addtocounter{subsection}{1}





\subsection{}

We have for every vertex~$i \in Q_0$ that~$P(i) = A \varepsilon_i$
It follows that
\[
    A
  = \bigoplus_{i \in Q_0} A \varepsilon_i
  = \bigoplus_{i \in Q_0} P(i)
\]
because~$Q_* = \coprod_{i \in Q_0} Q_* \varepsilon_i$, as~$Q_* \varepsilon_i$ is the set of all paths in~$Q$ starting in the vertex~$i$.
This shows that the~{\modules{$A$}} are direct summands of the free~{\module{$A$}}~$A$, and hence projective.





\addtocounter{subsection}{-2}
\subsection{}

We have for every~{\module{$A$}} an isomorphism of~{\kvss}
\[
  \Phi_2
  \colon
  \Hom_A(A, X)
  \to
  X \,,
  \quad
  f
  \mapsto
  f(1) \,.
\]
The~{\module{$A$}[$A$]} structure of~$A$ leads to a left~{\module{$A$}} structure on~$\Hom_A(A,X)$ given by
\[
    (a.f)(a')
  = f(a'a)
\]
for all~$a \in A$,~$f \in \Hom_A(A,X)$ and~$a' \in A$.
The above isomorphism of~{\kvs} is then an isomorphism of (left)~{\modules{$A$}} because
\[
    (a \cdot f)(1)
  = f(1 \cdot a)
  = f(a \cdot 1)
  = a \cdot f(1)
\]
for all~$a \in A$ and~$f \in \Hom_A(A,X)$.

The decomposition~$A = \bigoplus_{i \in Q_0} P(i)$ of~{\modules{$A$}} results in a decomposition
\[
  \Hom_A(A,X)
  =
  \Hom_A\left( \bigoplus_{i \in Q_0} P(i), X \right)
  \cong
  \bigoplus_{i \in Q_0} \Hom_A(P(i), X)
\]
of~{\kvss}.
The isomorphism
\[
  \Phi_1
  \colon
  \bigoplus_{i \in Q_0} \Hom_A(P(i), X)
  \to
  \Hom_A(A,X)
\]
is explicitely given by
\[
  \Phi_1((f_i)_{i \in Q_0})
  =
  \sum_{i \in Q_0} (f_i \circ \pi_i)
  =
  \sum_{i \in Q_0} \pi_i^*(f_i)
\]
where~$\pi_i \colon A \to P(i)$ denotes the projection along the decomposition~$A = \bigoplus_{i \in I}$.
This projection~$\pi_i$ is given by right multiplication with the idempotent~$\varepsilon_i$.
The complete isomorphism of~{\kvss}
\[
  \Phi
  \defined
  \Phi_2 \circ \Phi_1
  \colon
  \bigoplus_{i \in Q_0} \Hom_A(P(i), X)
  \to
  X
\]
is therefore given by
\[
    \Phi((f_i)_{i \in I})
  = \Phi_2(\Phi_1( (f_i)_{i \in I} ) )
  = \left( \sum_{i \in Q_0} \pi_i^*(f_i) \right)(1)
  = \sum_{i \in Q_0} f(\pi_i(1))
  = \sum_{i \in Q_0} f_i(\varepsilon_i)
\]
for every~$(f_i)_{i \in I} \in \bigoplus_{i \in I} \Hom_A(P(i), X)$.
The restriction
\[
  \Hom_A(P(j), X)
  \inclusion
  \bigoplus_{i \in Q_0} \Hom_A(P(i), X)
  \xlongto{\Phi}
  X
\]
is therefore given by
\[
  f
  \mapsto
  f(\varepsilon_i) \,.
\]
We note that
\[
  f(\varepsilon_i)
  =
  f(\varepsilon_i^2)
  =
  \varepsilon_i f(\varepsilon_i)
  \in
  \varepsilon_i X
  =
  X_i
\]
for every~$f \in \Hom_A(P(i), X)$.
This means that the isomorphism~$\Phi$ maps the direct summand~$\Hom_A(P(i), X)$ into the direct summand~$X_i$ of~$X$.
It follows that the resulting restriction
\[
  \Hom_A(P(i), X)
  \to
  X_i \,,
  \quad
  f
  \mapsto
  f(\varepsilon_i)
\]
is an isomorphism.





\addtocounter{subsection}{1}
\subsection{}

The quiver~$Q$ admits only finitely many paths because~$Q$ has only finitely many arrows and no oriented cycles.
It follows that~$kQ$ is {\fd}, and hence that every~$P(i)$ (which is a direct summand of~$kQ$) is {\fd}.
We also find that
\[
    \End_A(P(i))
  = \Hom_A(P(i),P(i))
  = P(i)_i
\]
has as a basis the set
\[
    Q_*(i,i)
  = \{
      p \in Q_*
    \suchthat
      s(p) = i = t(p)
    \}
  = \{
      \text{oriented cycles at~$i$}
    \}
    \cup
    \{ \varepsilon_i \}
  = \{
      \varepsilon_i
    \} \,.
\]
This shows that the~{\kalg}~$\End_A(P(i))$ is \dash{one}{dimensional}, which reveals to us that~$\End_A(P(i)) \cong k$ as~{\kalgs}.%
\footnote{Here we use that for every \dash{one}{dimensional}~{\kalg}~$B$ the map~$k \to B$,~$\lambda \to \lambda \cdot 1_B$ is an isomorphism of~{\kalgs}, and hence that~$B \cong k$ as~{\kalgs}. }





\subsection{}

If~$M$ is a decomposable~{\module{$A$}} then there exists a decomposition~$M = N \oplus P$ into two nonzero submodules~$N$ and~$P$ of~$M$.
The projection~$e \colon M \to M$ onto~$N$ along the decomposition~$M = N \oplus P$ is then an idempotent in~$\End_A(M)$ with
\[
  \im(e) = N
  \quad\text{and}\quad
  \ker(e) = P \,.
\]
It follows from~$N \neq 0$ that~$e \neq 0$, and from~$P \neq M$ that~$e \neq \id_M$.
We hence see that~$\End_A(M)$ contains a \dash{non}{trivial} idempotent if~$M$ is decomposable.%
\footnote{This construction does in fact give a bijection between the direct sum decompositions~$M = N \oplus P$ of~$M$ and the idempotents in the endomorphism ring~$\End_A(M)$.}

We have seen in the previous part of the exercise that~$\End_A(P(i)) = k$ as~{\kalgs}.
The field~$k$ contains no \dash{non}{trivial} idemponents, and hence~$P(i)$ is indecomposable.




