\section{}

We have for all~$i = 0, \dotsc, n$ and~$j = 0, \dotsc, n+1$ that
\begin{itemize}
  \item
    $f^{n+1}_j f^n_i = f^{n+1}_i f^n_{j-1}$ if~$j > i$ as both of these compositions are the unique order-preserving map~$\{0, \dotsc, n-1\} \to \{0, \dotsc, n+1\}$ whose image doesn’t contain~$i$ and~$j$, and
  \item
    $f^{n+1}_j f^n_i = f^{n+1}_{i+1} f^n_j$ if~$j \leq i$ as both of these compositions are the unique order-presvering map~$\{0, \dotsc, n-1\} \to \{0, \dotsc, n+1\}$ whose image doesn’t contain~$i+1$ and~$j$.
\end{itemize}





\subsection{}

We assume that~$C_n(A) = 0$ for all~$n < 0$, and accordingly~$d_n = 0$ for all~$n \leq 0$.
We then have that~$d_n d_{n+1} = 0$ for all~$n \leq 0$, and for all~$n > 0$ we have that
\begin{align*}
      d_n d_{n+1}
  &=  \left( \sum_{i=0}^n (-1)^i A(f^n_i) \right)
      \left( \sum_{j=0}^{n+1} (-1)^j A(f^{n+1}_j) \right) \\
  &=  \sum_{i=0}^n \sum_{j=0}^{n+1} (-1)^{i+j} A(f^{n+1}_j f^n_i) \\
  &=    \sum_{0 \leq i < j \leq n+1} (-1)^{i+j} A(f^{n+1}_j f^n_i)
      + \sum_{0 \leq j \leq i \leq n} (-1)^{i+j} A(f^{n+1}_j f^n_i) \,. \\
\end{align*}
We can rearrange the second sum as
\begin{align*}
   {}&  \sum_{0 \leq j \leq i \leq n} (-1)^{i+j} A(f^{n+1}_j f^n_i) \\
  ={}&  \sum_{0 \leq j \leq i \leq n} (-1)^{i+j} A(f^{n+1}_{i+1} f^n_j) \\
  ={}&  \sum_{0 \leq j < i \leq n+1} (-1)^{i-1+j} A(f^{n+1}_i f^n_j) \\
  ={}&  \sum_{0 \leq j < i \leq n+1} (-1)^{i-1+j} A(f^{n+1}_i f^n_j) \\
  ={}&  -\sum_{0 \leq j < i \leq n+1} (-1)^{i+j} A(f^{n+1}_i f^n_j)  \\
  ={}&  -\sum_{0 \leq i < j \leq n+1} (-1)^{i+j} A(f^{n+1}_j f^n_i) \,.
\end{align*}
With this we find that~$d_n d_{n+1} = 0$ also for~$n > 0$.





\subsection{}

We start by defining a cosimplical object~$\Sigma \colon \Delta \to \Top$ and then define the desired simplicial set as~$\Top(-,Y) \circ \Sigma \colon \Delta^\op \to \Set$.

For every~$n \geq 0$ let~$e_0, \dotsc, e_n$ be the standard basis of~$\Real^{n+1}$ and let
\[
  \Sigma (n)
  \defined
  \conv(e_0, \dotsc, e_n)
  \subseteq
  \Real^{n+1}
\]
be the standard~\dash{$n$}{simplex}, where~$\conv$ denotes the convex hull operator.%
\footnote{We avoid the usual notation~$\Delta^n$ for the standard simplex because there are already enough things named~$\Delta$.}
For all~$0 \leq m \leq n$ and every~$f \in \Delta(m,n)$ let~$\Sigma(f) \colon \Sigma(m) \to \Sigma(n)$ be the unique affine linear map with
\begin{equation}
  \label{action on corners}
    \Sigma(f)(e_i)
  = e_{f(i)}
\end{equation}
for every~$i = 0, \dotsc, m$;
more explicitely,
\[
    \Sigma(f)\left( \sum_{i=0}^m \lambda_i e_i \right)
  = \sum_{i=0}^m \lambda_i e_{f(i)}
\]
whenever~$\sum_{i=0}^n \lambda_i = 1$ with~$\lambda_i \geq 0$ for all~$i = 0, \dotsc, n$.
It follows from the description~$\eqref{action on corners}$ of~$\Sigma(f)$ that~$\Sigma(\id_n) = \id_{\Sigma(n)}$ for every~$n \geq 0$, and that~$\Sigma(gf) = \Sigma(g) \Sigma(f)$ for all composable morphisms~$f \colon k \to m$ and~$g \colon m \to n$ in~$\Delta$.
We have hence constructed a covariant functor~$\Sigma \colon \Delta \to \Top$, i.e.\ a cosimplical object in the category~$\Top$.

We set~$S \defined \Top(-,Y) \circ \Sigma \colon \Delta \to \Set$.
An object~$s \in S(n)$ is a continuous map~$s \colon \Delta^n \to Y$, which one might think about as an~\emph{\dash{$n$}{simplex} in~$Y$}.

Let~$\Csing(Y)$ be the singular chain complex of~$Y$.
We have from the above that
\[
  \Csing_n(Y) = FS(n) = C_n(FS)
\]
for every~$n \geq 0$.

Note that for eveny~$n \geq 0$ and~$i = 0, \dotsc, n$, the map~$\Sigma(f^n_i) \colon \Sigma(n-1) \to \Sigma(n)$ is precisely the inclusion of~$\Sigma(n-1)$ into~$\Sigma(n)$ as the~\dash{$i$}{th} face.
The differential of the singular chain complex~$\Csing_n(Y)$ is therefore given by
\[
    d^\sing_n(s)
  = \sum_{i=0}^n (-1)^i s \circ \Sigma(f^n_i)
\]
for every simplex~$s \in S(n)$.
The differential of the chain complex~$C_\bullet(FS)$ is also given by
\begin{align*}
      d_n(s)
  &=  \sum_{i=0}^n (-1)^i (FS)(f^n_i)(s)  \\
  &=  \sum_{i=0}^n (-1)^i (F \circ \Top(-,Y) \circ \Sigma)(f^n_i)(s)  \\
  &=  \sum_{i=0}^n (-1)^i F( \Sigma(f^n_i)^* )(s) \\
  &=  \sum_{i=0}^n (-1)^i \Sigma(f^n_i)^*(s)  \\
  &=  \sum_{i=0}^n (-1)^i s \circ \Sigma(f^n_i) \,.
\end{align*}
This shows that the two chain complexes~$\Csing$ and~$C_\bullet(FS)$ do have not only the same components but also the same differential, hence that they are the same.


