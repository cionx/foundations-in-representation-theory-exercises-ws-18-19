\section{}





\subsection{}

\begin{lemma}
  \label{functoriality of kernel}
  Let~$\Acat$ be an abelian category.
  \begin{enumerate}
    \item
      Let
      \[
        \begin{tikzcd}
            X
            \arrow{r}[above]{f}
            \arrow{d}[left]{\varphi}
          & Y
            \arrow{d}[right]{\varphi'}
          \\
            X'
            \arrow{r}[above]{f'}
          & Y'
        \end{tikzcd}
      \]
      be a commutative square in~$\Acat$.%
      \footnote{We may think about this commutative square a morphism~$(\varphi, \varphi') \colon f \to f'$ in the morphism category of~$\Acat$.}
      Let~$k \colon \ker(f) \to X$ and~$k' \colon \ker(f') \to X'$ be kernels of~$f$ and~$f'$.
      Then there exist a unique morphism~$\varphi'' \colon \ker(f) \to \ker(f'')$ that makes the following diagram commute:
      \[
        \begin{tikzcd}
            \ker(f)
            \arrow{r}[above]{k}
            \arrow[dashed]{d}[left]{\varphi''}
          & X
            \arrow{r}[above]{f}
            \arrow{d}[left]{\varphi}
          & Y
            \arrow{d}[right]{\varphi'}
          \\
            \ker(f')
            \arrow{r}[above]{k'}
          & X'
            \arrow{r}[above]{f'}
          & Y'
        \end{tikzcd}
      \]
    \item
      This induced morphism is functorial in the following sense:
      Let
      \[
        \begin{tikzcd}
            X
            \arrow{r}[above]{f}
            \arrow{d}[left]{\varphi}
          & Y
            \arrow{d}[right]{\varphi'}
          \\
            X'
            \arrow{r}[above]{f'}
            \arrow{d}[left]{\psi}
          & Y'
            \arrow{d}[right]{\psi''}
          \\
            X''
            \arrow{r}[above]{f''}
          & Y''
        \end{tikzcd}
      \]
      be a commutative diagram in~$\Acat$ and let
      \[
        k \colon \ker(f) \to X \,,
        \qquad
        k' \colon \ker(f') \to X' \,,
        \qquad
        k'' \colon \ker(f'') \to X''
      \]
      be kernels of~$f$,~$f'$ and~$f''$.
      Let~$\varphi'' \colon \ker(f) \to \ker(f')$ be the morphism induced by~$(\varphi,\varphi')$ and let~$\psi'' \colon \ker(f') \to \ker(f'')$ be the morphism induced by~$(\psi, \psi')$.
      Then the composition~$\psi'' \varphi'' \colon \ker(f) \to \ker(f'')$ is the morphism induced by~$(\psi \varphi, \psi' \varphi')$.
  \end{enumerate}
\end{lemma}

\begin{proof}
  \leavevmode
  \begin{enumerate}
    \item
      This follows from the universal property of the kernel~$k' \colon \ker(f') \to X'$ of~$f'$ because
      \[
          f' \varphi k
        = \varphi' f k
        = \varphi' \circ 0
        = 0 \,.
      \]
    \item
    We have the following commutative diagram:
      \[
        \begin{tikzcd}[sep = huge]
            \ker(f)
            \arrow{r}[above]{k}
            \arrow{d}[right]{\varphi'}
          & X
            \arrow{r}[above]{f}
            \arrow{d}[right]{\varphi}
          & Y
            \arrow{d}[right]{\varphi''}
          \\
            \ker(f')
            \arrow{r}[above, near start]{k'}
            \arrow{d}[right]{\psi'}
          & X'
            \arrow{r}[above, near start]{f'}
            \arrow{d}[right]{\psi}
          & Y'
            \arrow{d}[right]{\psi''}
          \\
            \ker(f'')
            \arrow{r}[above]{k''}
            \arrow[from=uu, dashed, bend right = 45, crossing over, "\psi' \varphi'" left, near start]
          & X''
            \arrow{r}[above]{f''}
            \arrow[from=uu, dashed, bend right = 45, crossing over, "\psi \varphi" left, near start]
          & Y''
            \arrow[from=uu, dashed, bend right = 45, crossing over, "\psi'' \varphi''" left, near start]
        \end{tikzcd}
      \]
      The commutativity of the subdiagram
      \[
        \begin{tikzcd}
            \ker(f)
            \arrow{r}[above]{k}
            \arrow[dashed]{d}[left]{\psi'' \varphi''}
          & X
            \arrow{r}[above]{f}
            \arrow{d}[left]{\psi \varphi}
          & Y
            \arrow{d}[right]{\psi' \varphi'}
          \\
            \ker(f'')
            \arrow{r}[above]{k''}
          & X''
            \arrow{r}[above]{f''}
          & Y'
        \end{tikzcd}
      \]
      shows that~$\psi'' \varphi''$ satisfies the defining property of the morphisms~$\ker(f) \to \ker(f'')$ induced by~$(\psi \varphi, \psi' \varphi')$.
    \qedhere
  \end{enumerate}
\end{proof}

For every~$n \in \Natural$ let~$k_n \colon \ker(f_n) \to C_n$ be a kernel of~$f_n \colon C_n \to D_n$.
It follows by \cref{functoriality of kernel} from the commutativity of the square
\[
  \begin{tikzcd}
      C_n
      \arrow{r}[above]{d_n}
      \arrow{d}[left]{f_n}
    & C_{n-1}
      \arrow{d}[right]{f_{n-1}}
    \\
      D_n
      \arrow{r}[above]{d_n}
    & D_{n-1}
  \end{tikzcd}
\]
that there exists a unique morphism~$d'_n \colon \ker(f_n) \to \ker(f_{n-1})$ that makes the following square commute:
\begin{equation}
  \label{induced differential}
  \begin{tikzcd}
      \ker(f_n)
      \arrow[dashed]{r}[above]{d'_n}
      \arrow{d}[left]{k_n}
    & \ker(f_{n-1})
      \arrow{d}[right]{k_{n-1}}
    \\
      C_n
      \arrow{r}[above]{d_n}
    & C_{n-1}
  \end{tikzcd}
\end{equation}
The composition~$d'_{n-1} d'_n$ is by \cref{functoriality of kernel} the unique morphism~$\ker(f_n) \to \ker(f_{n-2})$ that makes the following diagram commute:
\[
  \begin{tikzcd}
      \ker(f_n)
      \arrow[dashed, bend left]{rr}[above]{d'_n d'_{n-1}}
      \arrow[dashed]{r}[above]{d'_n}
      \arrow{d}[left]{k_n}
    & \ker(f_{n-1})
      \arrow[dashed]{r}[above]{d'_{n-1}}
      \arrow{d}[right]{k_{n-1}}
    & \ker(f_{n-2})
      \arrow{d}[right]{k_{n-2}}
    \\
      C_n
      \arrow{r}[above]{d_n}
      \arrow[bend right]{rr}[below]{0}
    & C_{n-1}
      \arrow{r}[above]{d_{n-1}}
    & C_{n-2}
  \end{tikzcd}
\]
The zero morphism also makes this diagram commute, hence~$d'_{n-1} d'_n = 0$.
This shows that~$\ker(f) = ((\ker(f_n))_{n \in \Integer}, (d'_n)_{n \in \Integer})$ is a chain complex.
The commutativity of the square~\eqref{induced differential} tells us furtheromer that~$k \defined (k_n)_{n \in \Integer}$ is a morphism of chain complexes~$k \colon \ker(f) \to C_\bullet$.

The composition~$f k$ vanishes because~$(fk)_n = f_n k_n = 0$ for every~$n \in \Integer$.
Suppose that~$g \colon B_\bullet \to C_\bullet$ is another morphism of chain complexs for which~$f g = 0$.
Then~$0 = (fg)_n = f_n g_n$ for every~$n \in \Integer$, and it follows from the universal property of the kernel~$k_n \colon \ker(f_n) \to C_n$ that there exist a unique morphism~$h_n \colon B_n \to \ker(f_n)$ that makes the following diagram commute:
\begin{equation}
  \label{constructing triangle}
  \begin{tikzcd}
      \ker(f_n)
      \arrow{r}[above]{k_n}
    & C_n
    \\
      B_n
      \arrow[dashed]{u}[left]{h_n}
      \arrow{ur}[below right]{g_n}
    & {}
  \end{tikzcd}
\end{equation}

Then~$h \defined (h_n)_{n \in \Integer}$ is a morphism of chain complexes: We have for every~$n \in \Integer$ the following diagram:
\begin{equation}
  \label{cheese block}
  \begin{tikzcd}
      B_n
      \arrow[bend left = 40]{rr}[above]{g_n}
      \arrow[dashed]{r}[above]{h_n}
      \arrow{d}[right]{d_n}
    & \ker(f_n)
      \arrow{r}[above]{k_n}
      \arrow{d}[right]{d'_n}
    & C_n
      \arrow{d}[right]{d_n}
    \\
      B_{n-1}
      \arrow[dashed]{r}[above]{h_{n-1}}
      \arrow[bend right = 40]{rr}[below]{g_{n-1}}
    & \ker(f_{n-1})
      \arrow{r}[above]{k_{n-1}}
    & C_{n-1}
  \end{tikzcd}
\end{equation}
The right square commutes because~$k$ is a morphism of chain complexes, the outer square commutes because~$g$ is a morphism of chain complexes and the upper and lower triangles commute by choice of~$h_n$ and~$h_{n-1}$.
It follows that the left square commutes, because
\[
    k_{n-1} d'_n h_n
  = d_n k_n h_n
  = d_n g_n
  = g_{n-1} d_n
  = k_{n-1} h_{n-1} d_n
\]
and hence~$d'_n h_n = h_{n-1} d_n$ since~$k_{n-1}$ is a monomorphism.

That the morphism~$h_n$ makes for every~$n \in \Integer$ the triangle~\eqref{constructing triangle} commute gives altogether that the morphism of chain complexes~$h \colon B_\bullet \to \ker(f)$ makes the following diagram commute:
\[
  \begin{tikzcd}
      \ker(f)
      \arrow{r}[above]{k}
    & C_\bullet
    \\
      B_\bullet
      \arrow[dashed]{u}[left]{h}
      \arrow{ur}[below right]{g}
    & {}
  \end{tikzcd}
\]
The morphism~$h$ is unique with this property:
If~$h' \colon B_\bullet \to \ker(f)$ is another morphim of chain complexes with~$k h' = g$ then~$k_n h'_n = g_n$ for every~$n \in \Integer$, and hence the triangle
\[
  \begin{tikzcd}
      \ker(f_n)
      \arrow{r}[above]{k_n}
    & C_n
    \\
      B_n
      \arrow[dashed]{u}[left]{h'_n}
      \arrow{ur}[below right]{g_n}
    & {}
  \end{tikzcd}
\]
commutes for every~$n \in \Integer$.
It then follows for every~$n \in \Integer$ from the uniqueness of~$h_n$ that~$h'_n = h_n$, and hence overall~$h' = h$.

This shows altogether that the morphism of chain complexes~$k \colon \ker(f) \to C_\bullet$ is a kernel of~$f$.
This explicit construction of the kernel also shows that kernels in~$\Ch_\bullet(\Acat)$ can be computed degree-wise.

\begin{remark}
  We similary find that the cokernel of~$f$ is given by a chain complex
  \[
      \coker(f)
    = ( (\coker(f_n))_{n \in \Integer}, (d''_n)_{n \in \Integer} )
  \]
  together with a morphism of chain complexes~$c \colon D_\bullet \to \coker(f)$ such that
  \begin{itemize}
    \item
      the morphism $c_n \colon D_n \to \coker(f_n)$ is for every~$n \in \Integer$ a cokernel of the morphism~$f_n \colon C_n \to D_n$, and
    \item
      the differential~$d''_n \colon \coker(f_n) \to \coker(f_{n-1})$ is for every~$n \in \Integer$ the unique morphism~$\coker(f_n) \to \coker(f_{n-1})$ that makes the following square commute:
      \[
        \begin{tikzcd}
            D_n
            \arrow{r}[above]{d_n}
            \arrow{d}[left]{c_n}
          & D_{n-1}
            \arrow{d}[right]{c_{n-1}}
          \\
            \coker(f_n)
            \arrow[dashed]{r}[below]{d''_n}
          & \coker(f_{n-1})
        \end{tikzcd}
      \]
    \item
      If~$g \colon D_\bullet \to E_\bullet$ is another morphism of chain complexes with~$g f = 0$, then the unique morphism of chain complexes~$h \colon \coker(f) \to E_\bullet$ that makes the triangle
      \[
        \begin{tikzcd}
            D_\bullet
            \arrow{r}[above]{c}
            \arrow{dr}[below left]{g}
          & \coker(f)
            \arrow[dashed]{d}[right]{h}
          \\
            {}
          & E_\bullet
        \end{tikzcd}
      \]
      commute can be computed degree-wise, i.e.\ the component~$h_n \colon \coker(f_n) \to E_n$ is for every~$n \in \Integer$ the unique morphism~$\coker(f_n) \to E_n$ that makes the following triangle commute:
      \[
        \begin{tikzcd}
            D_n
            \arrow{r}[above]{c}
            \arrow{dr}[below left]{g_n}
          & \coker(f_n)
            \arrow[dashed]{d}[right]{h_n}
          \\
            {}
          & E_n
        \end{tikzcd}
      \]
  \end{itemize}
  We can similarly calculate the image and coimage of~$f$ degree-wise
\end{remark}





\subsection{}

Consider the case~$\Acat = \Modl{\Integer}$ and the acylic chain complexes
\begin{align*}
      C_\bullet
  &=  (
        \dotsb
        \to
        0
        \to
        2 \Integer
        \xlongto{i}
        \Integer
        \xlongto{p}
        \Integer/2
        \to
        0
        \to
        \dotsb
      ) \,,
  \\
      D_\bullet
  &=  (
        \dotsb
        \to
        0
        \to
        \Integer
        \xlongto{\id}
        \Integer
        \to
        0
        \to
        0
        \dotsb
      ) \,,
\end{align*}
where~$i$ denotes the inclusion and~$p$ the canonical projection.
We have0a morphism of chain complexes~$f \colon C_\bullet \to D_\bullet$ that is given by the following commutative ladder:
\[
  \begin{tikzcd}
      \dotsb
      \arrow{r}
    & 0
      \arrow{r}
      \arrow[dashed]{d}
    & 2\Integer
      \arrow{r}[above]{i}
      \arrow[dashed]{d}[right]{i}
    & \Integer
      \arrow{r}[above]{p}
      \arrow[dashed]{d}{\id_\Integer}
    & \Integer/2
      \arrow{r}
      \arrow[dashed]{d}
    & 0
      \arrow{r}
      \arrow[dashed]{d}
    & \dotsb
    \\
      \dotsb
      \arrow{r}
    & 0
      \arrow{r}
    & \Integer
      \arrow{r}[above]{\id_\Integer}
    & \Integer
      \arrow{r}
    & 0
      \arrow{r}
    & 0
      \arrow{r}
    & \dotsb
  \end{tikzcd}
\]
We can compute the kernel, cokernel and image of~$f$ degree-wise, and hence find that
\begin{align*}
      \ker(f)
  &=  (
        \dotsb
        \to
        0
        \to
        0
        \to
        0
        \to
        \Integer/2
        \to
        0
        \to
        \dotsb
      ) \,,
  \\
      \coker(f)
  &=  (
        \dotsb
        \to
        0
        \to
        \Integer/2
        \to
        0
        \to
        0
        \to
        0
        \to
        \dotsb
      ) \,,
  \\
      \im(f)
  &=  (
        \dotsb
        \to
        0
        \to
        2\Integer
        \xlongto{i}
        \Integer
        \to
        0
        \to
        0
        \to
        \dotsb
      ) \,.
\end{align*}
None of those three chain complexes is acyclic.




