\section{}





\subsection{}

The given functor~$F$ is represented by the polynomial ring~$k[X_1, \dotsc, X_n]$.
Indeed, there exist for every commutative~{\kalg}~$B$ a bijection
\[
          \eta_B
  \colon  \Hom_{\kCommAlg}(k[X_1, \dotsc, X_n], B)
  \to     B^n \,,
  \quad   f
  \mapsto ( f(X_1), \dotsc, f(X_n) )
\]
by the universal property of the polynomial ring.
These bijections are natural and hence yield an isomorphism of functors~$\eta \colon \Hom_{\kCommAlg}(k[X_1, \dotsc, X_n], -) \to F$.





\addtocounter{subsection}{1}





\subsection{}

An~\dash{$A$}{valued}~\dash{$(n \times n)$}{matrix}~$M$ is invertible if and only if its determinant~$\det(M) \in A$ is a unit, i.e.\ if there exists an element~$a \in A$ with~$\det(M) a = 1$;
the value~$a$ is then uniquely determined as~$a = \det(M)^{-1}$.
It follows that we have bijections
\begin{align*}
       {}&  \GL_n(A)  \\
      ={}&  \{
              M \in \mat{n}{A}
              \suchthat
            \exists a \in A : \det(M) a  = 1
            \} \\
  \cong{}&  \{
              (M,a) \in \mat{n}{A} \times A
            \suchthat
              \det(M) a = 1
            \} \\
  \cong{}&  \{
              (a_{11}, \dotsc, a_{nn}, a) \in A^{n^2 + 1}
            \suchthat
              \det(a_{11}, \dotsc, a_{nn}) a = 1
            \}  \\
      ={}&  \{
              (a_{11}, \dotsc, a_{nn}, a) \in A^{n^2 + 1}
            \suchthat
              \det(a_{11}, \dotsc, a_{nn}) a - 1 = 0
            \}  \\
  \cong{}&  \Hom_{\kCommAlg}(k[X_{11}, \dotsc, X_{nn}, Y]/({\det} \cdot Y - 1), A)  \\
  \cong{}&  \Hom_{\kCommAlg}(k[X_{11}, \dotsc, X_{nn}]_{\det}, A)
\end{align*}
where we denote by abuse of notation with~$\det \in k[X_{11}, \dotsc, X_{nn}]$ the polynomial
\[
    \det
  = \sum_{\sigma \in S_n} \sign(\sigma) X_{1 \sigma(1)} \dotsm X_{n \sigma(n)} \,.
\]
The above bijection is (in reverse order) overall given by
\begin{align*}
            \eta_A
   \colon   \Hom_{\kCommAlg}(k[X_{11}, \dotsc, X_{nn}]_{\det}, A)
  &\to      \GL_n(A) \,,
  \\
            f
  &\mapsto  \begin{bmatrix}
              f(X_{11}) & \cdots  & f(X_{1n}) \\
              \vdots    & \ddots  & \vdots    \\
              f(X_{n1}) & \cdots  & f(X_{nn})
            \end{bmatrix} \,,
\end{align*}
where we identify~$k[X_{11}, \dotsc, X_{nn}]$ with a subalgebra of~$k[X_{11}, \dotsc, X_{nn}]_{\det}$ via the canonical homomorphism~$f \mapsto f/1$ (which is possible because~$k[X_{11}, \dotsc, X_{nn}]$ is an integral domain).
These bijections are natural and hence give rise to a natural isomorphism~$\eta \colon \Hom_{\kCommAlg}(k[X_{11}, \dotsc, X_{nn}]_{\det}, -) \to F$.
The functor~$F$ is therefore representable by~$k[X_{11}, \dotsc, X_{nn}]_{\det}$.




