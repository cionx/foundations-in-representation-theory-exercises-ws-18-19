\section{}

A homomorphism~$f \colon X_{(a,b)} \to X_{(c,d)}$ is the same as a pair~$f = (\lambda, \mu)$ consisting of scalars~$\lambda, \mu \in k$ such that the diagrams
\[
  \begin{tikzcd}
      k
      \arrow{r}[above]{a}
      \arrow{d}[left]{\lambda}
    & k
      \arrow{d}[right]{\mu}
    \\
      k
      \arrow{r}[above]{c}
    & k
  \end{tikzcd}
  \qquad\text{and}\qquad
  \begin{tikzcd}
      k
      \arrow{r}[above]{b}
      \arrow{d}[left]{\lambda}
    & k
      \arrow{d}[right]{\mu}
    \\
      k
      \arrow{r}[above]{d}
    & k
  \end{tikzcd}
\]
commute, i.e.\ such that
\[
  \left\{
    \begin{aligned}
      \mu a &= c \lambda \,, \\
      \mu b &= d \lambda \,,
    \end{aligned}
  \right.
  \iff
  \begin{bmatrix}
    c & -a  \\
    d & -b
  \end{bmatrix}
  \begin{bmatrix}
    \lambda \\
    \mu
  \end{bmatrix}
  = 0
  \iff
  \begin{bmatrix}
    \lambda \\
    \mu
  \end{bmatrix}
  \in \ker
  \begin{bmatrix}
    c & -a  \\
    d & -b
  \end{bmatrix} \,.
\]
We now distinguish between some cases:
\begin{itemize}
  \item
    If~$(a,b)$ and~$(c,d)$ are linearly independent then the matrix~%
    $
      \begin{bsmallmatrix}
        c & -a  \\
        d & -b
      \end{bsmallmatrix}
    $ %
    is invertible and it follows that~$\Hom(X_{(a,b)}, X_{(c,d)}) = 0$.
  \item
    If~$(a,b) = (0,0)$ but~$(c,d) \neq (0,0)$ then~$\Hom(X_{(a,b)}, X_{(c,d)}) = \{ (0,\mu) \suchthat \mu \in k \}$.
  \item
    If~$(a,b) \neq (0,0)$ but~$(c,d) = (0,0)$ then~$\Hom(X_{(a,b)}, X_{(c,d)}) = \{ (\lambda, 0) \suchthat \lambda \in k \}$.
  \item
    If~$(a,b), (c,d) \neq (0,0)$ are linearly dependent then~$(c,d)$ is a nonzero scalar multiple of~$(a,b)$ and the space~$\Hom(X_{(a,b)}, X_{(c,d)})$ is \dash{one}{dimensional}
    We further distinguish between two non-exclusive cases:
    \begin{itemize}
      \item
        If~$a \neq 0$ then also~$c \neq 0$ (because~$(c,d)$ is a nonzero scalar multiple of~$(a,b)$) and~$\Hom(X_{(a,b)}, X_{(c,d)}) = \{ \kappa(a,c) \suchthat \kappa \in k \}$.
      \item
        If~$b \neq 0$ then also~$d \neq 0$, and~$\Hom(X_{(a,b)}, X_{(c,d)}) = \{ \kappa(b,d) \suchthat \kappa \in k \}$.
    \end{itemize}
  \item
    If~$(a,b) = (c,d) = (0,0)$ then~$\Hom(X_{(a,b)}, X_{(c,d)}) = \{ (\lambda, \mu) \suchthat \lambda, \mu \in k \}$.
\end{itemize}
The two representations~$X_{(a,b)}$ and~$X_{(c,d)}$ are isomorphic if and only if there exists some~$(\lambda, \mu) \in \Hom(X_{(a,b)}, X_{(c,d)})$ with both~$\lambda \neq 0$ and~$\mu \neq 0$.
We see from the discussion above that this happens in only in two cases:
\begin{itemize}
  \item
    If~$(a,b)$ and~$(c,d)$ are both nonzero but linearly dependent.
  \item
    If~$(a,b) = (c,d) = (0,0)$.
    In this case it already holds that~$X_{(a,b)} = X_{(0,0)} = X_{(c,d)}$.
\end{itemize}




