\section{}





\subsection{}

The~{\module{$k$}} of~$R$ comes from identifying~$R$ with~$A \times X \times B$ via the bijection
\[
          R
  \to     A \times X \times B \,,
  \quad   \begin{bmatrix}
            a & x \\
            0 & b
          \end{bmatrix}
  \mapsto (a,x,b) \,.
\]
The proposed multiplication is {\welldef} because
\[
  \begin{bmatrix}
    a_1 & x_1 \\
    0   & b_1
  \end{bmatrix}
  \begin{bmatrix}
    a_2 & x_2 \\
    0   & b_2
  \end{bmatrix}
  =
  \begin{bmatrix}
    a_1 a_2 & a_1 x_2 + x_1 b_2 \\
          0 & b_1 b_2
  \end{bmatrix}
  \in
  R \,.
\]
The multiplication is associative because
\begin{align*}
  {}&
  \left(
    \begin{bmatrix}
      a_1 & x_1 \\
      0   & b_1
    \end{bmatrix}
    \begin{bmatrix}
      a_2 & x_2 \\
      0   & b_2
    \end{bmatrix}
  \right)
  \begin{bmatrix}
    a_3 & x_3 \\
    0   & b_3
  \end{bmatrix}
  \\
  ={}&
  \begin{bmatrix}
    a_1 a_2 & a_1 x_2 + x_1 b_2 \\
    0       & b_1 b_2
  \end{bmatrix}
  \begin{bmatrix}
    a_3 & x_3 \\
    0   & b_3
  \end{bmatrix}
  \\
  ={}&
  \begin{bmatrix}
    a_1 a_2 a_3 & a_1 a_2 x_3 + a_1 x_2 b_3 + x_1 b_2 b_3 \\
    0           & b_1 b_2 b_3
  \end{bmatrix}
  \\
  ={}&
  \begin{bmatrix}
      a_1 & x_1 \\
      0   & b_1
  \end{bmatrix}
  \begin{bmatrix}
    a_2 a_3 & a_2 x_3 + x_2 b_3 \\
    0       & b_2 b_3
  \end{bmatrix}
  \\
  ={}&
  \begin{bmatrix}
    a_1 & x_1 \\
    0   & b_1
  \end{bmatrix}
  \left(
    \begin{bmatrix}
      a_2 & x_2 \\
      0   & b_2
    \end{bmatrix}
    \begin{bmatrix}
      a_3 & x_3 \\
      0   & b_3
    \end{bmatrix}
  \right) \,.
\end{align*}
The multiplication is distributive on the left because
\begin{align*}
  {}&
  \left(
    \begin{bmatrix}
      a_1 & x_1 \\
      0   & b_1
    \end{bmatrix}
    +
    \begin{bmatrix}
      a_2 & x_2 \\
      0   & b_2
    \end{bmatrix}
  \right)
  \begin{bmatrix}
    a_3 & x_3 \\
    0   & b_3
  \end{bmatrix}
  \\
  ={}&
  \begin{bmatrix}
    a_1 + a_2 & x_1 + x_2 \\
    0         & b_1 + b_2
  \end{bmatrix}
  \begin{bmatrix}
    a_3 & x_3 \\
    0   & b_3
  \end{bmatrix}
  \\
  ={}&
  \begin{bmatrix}
    (a_1 + a_2) a_3 & (a_1 + a_2) x_3 + (x_1 + x_2) b_3 \\
    0               & (b_1 + b_2) b_3
  \end{bmatrix}
  \\
  ={}&
  \begin{bmatrix}
    a_1 a_3 + a_2 a_3 & a_1 x_3 + a_2 x_3 + x_1 b_3 + x_2 b_3 \\
    0                 & b_1 b_3 + b_2 b_3
  \end{bmatrix}
  \\
  ={}&
  \begin{bmatrix}
    a_1 a_3 & a_1 x_3 + x_1 b_3 \\
    0       & b_1 b_3
  \end{bmatrix}
  +
  \begin{bmatrix}
    a_2 a_3 & a_2 x_3 + x_2 b_3 \\
    0       & b_2 b_3
  \end{bmatrix}
  \\
  ={}&
  \begin{bmatrix}
    a_1 & x_1 \\
    0   & b_1
  \end{bmatrix}
  \begin{bmatrix}
    a_3 & x_3 \\
    0   & b_3
  \end{bmatrix}
  +
  \begin{bmatrix}
    a_2 & x_2 \\
    0   & b_2
  \end{bmatrix}
  \begin{bmatrix}
    a_3 & x_3 \\
    0   & b_3
  \end{bmatrix} \,.
\end{align*}
The distributivity on the right can be shown in the same way.
The multiplication is already~{\kbil} because
\begin{align*}
  {}&
  \left(
    \lambda
    \begin{bmatrix}
      a_1 & x_1 \\
      0   & b_1
    \end{bmatrix}
  \right)
  \begin{bmatrix}
    a_2 & x_2 \\
    0   & b_2
  \end{bmatrix}
  \\
  ={}&
  \begin{bmatrix}
    \lambda a_1 & \lambda x_1 \\
    0           & \lambda b_1
  \end{bmatrix}
  \begin{bmatrix}
    a_2 & x_2 \\
    0   & b_2
  \end{bmatrix}
  \\
  ={}&
  \begin{bmatrix}
    \lambda a_1 a_2 & \lambda a_1 x_2 + \lambda x_1 b_2 \\
    0               & \lambda b_1 b_2
  \end{bmatrix}
  \\
  ={}&
  \lambda
  \begin{bmatrix}
    a_1 a_2 & a_1 x_2 + x_1 b_2 \\
    0       & b_1 b_2
  \end{bmatrix}
  \\
  ={}&
  \lambda
  \left(
    \begin{bmatrix}
      a_1 & x_1 \\
      0   & b_1
    \end{bmatrix}
    \begin{bmatrix}
      a_2 & x_2 \\
      0   & b_2
    \end{bmatrix}
  \right) \,,
\end{align*}
und similarly
\[
  \begin{bmatrix}
    a_1 & x_1 \\
    0   & b_1
  \end{bmatrix}
  \left(
    \lambda
    \begin{bmatrix}
      a_1 & x_1 \\
      0   & b_1
    \end{bmatrix}
  \right)
  =
  \dotsb
  =
  \lambda
  \left(
    \begin{bmatrix}
      a_1 & x_1 \\
      0   & b_1
    \end{bmatrix}
    \begin{bmatrix}
      a_2 & x_2 \\
      0   & b_2
    \end{bmatrix}
  \right) \,.
\]
The multiplicative unit of~$R$ is given by the identity matrix because
\[
  \begin{bmatrix}
    1 & 0 \\
    0 & 1
  \end{bmatrix}
  \begin{bmatrix}
    a & x \\
    0 & b
  \end{bmatrix}
  =
  \begin{bmatrix}
    a & x \\
    0 & b
  \end{bmatrix}
  =
  \begin{bmatrix}
    a & x \\
    0 & b
  \end{bmatrix}
  \begin{bmatrix}
    1 & 0 \\
    0 & 1
  \end{bmatrix} \,.
\]
This shows altogether that~$R$ together with the given~{\module{$k$}} structure and multiplication is a~{\kalg}.





\subsection{}

The~{\klin} map
\[
    X \tensor_k N
  \xlongto{ \lambda \tensor \id_N }
    \Hom_k(N, M) \tensor_k N
  \xlongto[f \tensor n \mapsto f(n)]{ \text{eval} }
    M
\]
corresponds to a~{\kbil} map
\begin{equation}
  \label{action of X on N}
          X \times M
  \to     N \,,
  \quad   (x,n)
  \mapsto \lambda(x)(n) \,,
\end{equation}
which we might think about as an action of~$X$ on~$N$ to~$M$.
We hence write
\[
            x \cdot n
  \defined  \lambda(x)(n)
\]
for all~$x \in X$,~$n \in N$.
It follows from the~\dash{$k$}{bilinearity} of~\eqref{action of X on N} that
\[
    (x_1 + x_2) n
  = x_1 n + x_2 n \,,
  \quad
    x (n_1 + n_2)
  = x n_1 + x n_2 \,,
  \quad
    (\lambda x) n
  = \lambda (xn)
  = x (\lambda n)
\]
for all~$x, x_1, x_2 \in X$,~$n, n_1, n_2 \in N$,~$\lambda \in k$.
It further follows from~$\lambda$ being a homomorphism of~{\modules{$A$}[$B$]} that
\begin{gather}
  \label{asso a}
    (a x) n
  = \lambda(ax)(n)
  = (a \lambda(x))(n)
  = a \lambda(x)(n)
  = a (x n)
\shortintertext{and}
  \label{asso b}
    (x b) n
  = \lambda(xb)(n)
  = (\lambda(x)b)(n)
  = \lambda(x)(bn)
  = x (b n) \,.
\end{gather}

As the underlying~{\module{$k$}} of~$F(M,N,\lambda)$ we choose~$M \dsum N$, and write the elements of~$F(M,N,\lambda)$ as column vectors
\[
  \vect{m \\ n}
\]
with~$m \in M$,~$n \in N$.
The action of~$R$ on~$F(M,N,\lambda)$ is given by
\[
    \begin{bmatrix}
      a & x \\
      0 & b
    \end{bmatrix}
    \vect{m \\ n}
  = \vect{ am + xn \\ bn } \,.
\]
This action is distributive on the left because
\begin{align*}
  {}&
  \left(
    \begin{bmatrix}
      a_1 & x_1 \\
      0   & b_1
    \end{bmatrix}
    +
    \begin{bmatrix}
      a_2 & x_2 \\
      0   & b_2
    \end{bmatrix}
  \right)
  \vect{m \\ n}
  \\
  ={}&
  \begin{bmatrix}
    a_1 + a_2 & x_1 + x_2 \\
    0         & b_1 + b_2
  \end{bmatrix}
  \vect{m \\ n}
  \\
  ={}&
  \vect{ (a_1 + a_2) m + (x_1 + x_2) n \\ (b_1 + b_2) n  }
  \\
  ={}&
  \vect{ a_1 m + a_2 m + x_1 n + x_2 n \\ b_1 n + b_2 n }
  \\
  ={}&
    \vect{ a_1 m + x_1 n \\ b_1 n }
  + \vect{ a_2 m + x_2 n \\ b_2 n }
  \\
  ={}&
  \begin{bmatrix}
    a_1 & x_1 \\
    0   & b_1
  \end{bmatrix}
  \vect{m \\ n}
  +
  \begin{bmatrix}
    a_2 & x_2 \\
    0   & b_2
  \end{bmatrix}
  \vect{m \\ n} \,,
\end{align*}
and distributive on the right because
\begin{align*}
  {}&
  \begin{bmatrix}
    a & x \\
    0 & b
  \end{bmatrix}
  \left(
      \vect{ m_1 \\ n_1 }
    + \vect{ m_2 \\ n_2 }
  \right)
  \\
  ={}&
  \begin{bmatrix}
    a & x \\
    0 & b
  \end{bmatrix}
  \vect{ m_1 + m_2 \\ n_1 + n_2 }
  \\
  ={}&
  \vect{ a (m_1 + m_2) + x (n_1 + n_2) \\ b (n_1 + n_2) }
  \\
  ={}&
  \vect{ a m_1 + a m_2 + x n_1 + x n_2 \\ b n_1 + b n_2 }
  \\
  ={}&
    \vect{ a m_1 + x n_1 \\ b n_1 }
  + \vect{ a m_2 + x n_2 \\ b n_2 }
  \\
  ={}&
  \begin{bmatrix}
    a & x \\
    0 & b
  \end{bmatrix}
  \vect{ m_1 \\ n_1 }
  +
  \begin{bmatrix}
    a & x \\
    0 & b
  \end{bmatrix}
  \vect{ m_2 \\ n_2 } \,.
\end{align*}
The multiplicaiton is already~{\kbil} because
\begin{align*}
  \left(
    \mu
    \begin{bmatrix}
      a & x \\
      0 & b
    \end{bmatrix}
  \right)
  \vect{ m \\ n }
  =
  \begin{bmatrix}
    \mu a & \mu x \\
    0     & \mu b
  \end{bmatrix}
  \vect{ m \\ n }
  =
  \vect{\mu a m + \mu x n \\ \mu b n}
  &=
  \mu \vect{a m + x n \\ b n}
  \\
  &=
  \mu
  \left(
    \begin{bmatrix}
      a & x \\
      0 & b
    \end{bmatrix}
    \vect{ m \\ n }
  \right)
\end{align*}
and similarly
\begin{align*}
  \begin{bmatrix}
    a & x \\
    0 & b
  \end{bmatrix}
  \left(
    \mu
    \vect{m \\ n}
  \right)
  =
  \begin{bmatrix}
    a & x \\
    0 & b
  \end{bmatrix}
  \vect{\mu m \\ \mu n}
  =
  \vect{ \mu am + \mu xn \\ \mu bn }
  &=
  \mu 
  \vect{am + xn \\ bn}
  \\
  &=
  \mu
  \left(
    \begin{bmatrix}
      a & x \\
      0 & b
    \end{bmatrix}
    \vect{m \\ n}
  \right)
\end{align*}
The multiplication is associative because
\begin{align*}
  {}&
  \begin{bmatrix}
    a_1 & x_1 \\
    0   & b_1
  \end{bmatrix}
  \left(
    \begin{bmatrix}
      a_2 & x_2 \\
      0   & b_2
    \end{bmatrix}
    \vect{m \\ n}
  \right)
  \\
  ={}&
  \begin{bmatrix}
    a_1 & x_1 \\
    0   & b_1
  \end{bmatrix}
  \vect{ a_2 m + x_2 n \\ b_2 n }
  \\
  ={}&
  \vect{ a_1 a_2 m + a_1 x_2 n + x_1 b_2 n \\ b_1 b_2 n }
  \\
  ={}&
  \vect{ (a_1 a_2) m + (a_1 x_2 + x_1 b_2) n \\ (b_1 b_2) n }
  \\
  ={}&
  \begin{bmatrix}
    a_1 + a_2 & a_1 x_2 + x_1 b_2 \\
    0         & b_1 + b_2
  \end{bmatrix}
  \vect{m \\ n}
  \\
  ={}&
  \left(
    \begin{bmatrix}
      a_1 & x_1 \\
      0   & b_1
    \end{bmatrix}
    \begin{bmatrix}
      a_2 & x_2 \\
      0   & b_2
    \end{bmatrix}
  \right)
  \vect{m \\ n} \,,
\end{align*}
where we use the associativity of~\eqref{asso a} and~\eqref{asso b}.
It also holds that
\[
  \begin{bmatrix}
    1 & 0 \\
    0 & 1
  \end{bmatrix}
  \vect{m \\ n}
  =
  \vect{1 \cdot m + 0 \cdot n \\ 1 \cdot n}
  =
  \vect{m \\ n} \,.
\]
Altogether we have defined on~$F(M,N,\lambda)$ the structure of a left~{\module{$R$}}.





\subsection{}

Let~$(M, N, \lambda)$ and~$(M', N', \lambda')$ be two triples consisting of left~{\modules{$A$}}~$M$ and~$M'$, left~{\modules{$B$}}~$N$ and~$N'$, and homomorphism of~{\modules{$A$}[$B$]}~$\lambda \colon X \to \Hom_k(N, M)$ and~$\lambda' \colon X \to \Hom_k(N', M')$.
We define a \emph{morphism}~$f \colon (M, N, \lambda) \to (M', N', \lambda')$ to be a pair~$f = (f_1, f_2)$ consisting of a homomorphism of left~{\modules{$A$}}~$f_1 \colon M \to M'$ and a homomorphism of left~{\modules{$B$}}~$f_2 \colon N \to N'$, subject to the relation that
\[
    f_1 \circ \lambda(x)
  = \lambda'(x) \circ f_2
\]
for all~$x \in X$; this means for the actions of~$X$ on~$N$ to~$M$ and on~$N'$ to~$M'$ that
\[
    f_1(xn)
  = x f_2(n)
\]
for all~$x \in X$,~$n \in N$.
This may be represented pictorially by the following \enquote{commutative diagram}:
\[
  \begin{tikzcd}[column sep = large]
      M
      \arrow{d}[left]{f_1}
    & N
      \arrow{l}[above]{x \cdot (-)}
      \arrow{d}[right]{f_2}
    \\
      M'
    & N'
      \arrow{l}[above]{x \cdot (-)}
  \end{tikzcd}
\]
The \emph{composition} of two such morphisms
\begin{align*}
  f   \colon (M, N, \lambda)    &\to  (M', N', \lambda') \,,  \\
  f'  \colon (M', N', \lambda') &\to  (M'', N'', \lambda'')
\end{align*}
is given by
\[
    f' \circ f
  = (f'_1, f'_2) \circ (f_1, f_2)
  = (f'_1 \circ f_1, f_2' \circ f_2) \,.
\]
This is again a morphism because
\[
    f'_1 \circ f_1 \circ \lambda(x)
  = f'_1 \circ \lambda'(x) \circ f_2
  = \lambda''(x) \circ f'_2 \circ f_2
\]
for every~$x \in X$.
The identity morphism of~$(M, N, \lambda)$ is given by~$\id_{(M,N,\lambda)} = (\id_M, \id_N)$;
this is a morphisms because
\[
    {\id_M} \circ \lambda(x)
  = \lambda(x)
  = \lambda(x) \circ \id_N
\]
for every~$x \in X$;
it holds for every morphism~$f \colon (M,N,\lambda) \to (M',N',\lambda')$ that~$f \circ \id = f$, and for every morphisms~$f \colon (M',N',\lambda') \to (M,N,\lambda)$ that~$\id \circ f = f$.

The construction~$F$ is functorial:
Every morphism~$f \colon (M, N, \lambda) \to (M', N', \lambda')$ induces a map
\[
          F(f)
  \colon  F(M,N,\lambda)
  \to     F(M',N',\lambda') \,,
  \quad   \vect{m \\ n}
  \mapsto \vect{f_1(m) \\ f_2(n)} \,;
\]
we hence have that $F(f) = f_1 \dsum f_2$.
The additive map~$F(f)$ is a homomorphism of left~{\modules{$R$}} because
\begin{align*}
  F(f)
  \left(
    \begin{bmatrix}
      a & x \\
      0 & b
    \end{bmatrix}
    \vect{m \\ n}
  \right)
  &=
  F(f)
  \left(
    \vect{ am + xn \\ bn }
  \right)
  =
  \vect{ f_1(am + xn) \\ f_2(bn) }
  \\
  &=
  \vect{ f_1(am) + f_1(xn) \\ f_2(bn) }
  =
  \vect{ a f_1(m) + x f_2(n) \\ b f_2(n) }
  \\
  &=
  \begin{bmatrix}
    a & x \\
    0 & b
  \end{bmatrix}
  \vect{ f_1(m) \\ f_2(n) }
  =
  \begin{bmatrix}
    a & x \\
    0 & b
  \end{bmatrix}
  F(f)
  \left(
    \vect{m \\ n}
  \right) \,.
\end{align*}
It furthermore holds for any two composable morphisms~$f \colon (M, N, \lambda) \to (M', N', \lambda')$ and~$f' \colon (M', N', \lambda') \to (M'', N'', \lambda'')$ that
\begin{align*}
      F(f') \circ F(f)
  &=  (f'_1 \dsum f'_2) \circ (f_1 \dsum f_2)
   =  (f'_1 \circ f_1) \dsum (f'_2 \circ f_2) \\
  &=  (f' \circ f)_1 \dsum (f' \circ f)_2
   =  F(f' \circ f) \,,
\end{align*}
and it holds that
\[
  F(\id_{(M, N, \lambda)})
  =
  \id_M \dsum \id_N
  =
  \id_{M \dsum N}
  =
  \id_{F(M, N, \lambda)} \,.
\]
This shows altogether the claimed functoriality of~$F$.




















