\section{}





\subsection{}

The path algebra of the quiver~$Q$ is given by~$kQ \cong k[x]$, and representations of~$Q$ over~$k$ are therefore \enquote{the same} as~{\modules{$k[x]$}}.
To be more precise, a representation
\[
  \begin{tikzcd}
    V
    \arrow[loop right]{}[right]{\varphi}
  \end{tikzcd}
\]
of~$Q$ corresponds to the~{\module{$k[x]$}} whose underlying~{\kvs} is given by~$V$ and for which the action of~$x$ on~$V$ is given by~$\varphi$.
It follows from the classification of finitely generated~{\modules{$k[x]$}} and~$k$ being algebraically closed that the {\fd} indecomposable~{\modules{$k[x]$}} are up to isomorphism precisely those of the form
\[
  k[x]/(x-\lambda)^n
\]
with~$\lambda \in k$ and~$n \geq 1$, and that these representations are pairwise nonisomorphic.
With respect to the basis~$1, (x-\lambda), \dotsc, (x-\lambda)^{n-1}$ of~$k[x]/(x-\lambda)^n$ the action of~$x$ is given by the Jordan block matrix
\[
  \begin{bmatrix}
      \lambda
    & {}
    & {}
    & {}
    \\
      1
    & \ddots
    & {}
    & {}
    \\
      {}
    & \ddots
    & \ddots
    & {}
    \\
      {}
    & {}
    & 1
    & \lambda
  \end{bmatrix} \,.
\]
This shows altogether that the representations
\[
  \begin{tikzcd}[ampersand replacement = \&]
    k^n
    \arrow[loop right]{}[right]{
      \begin{bsmallmatrix}
          \lambda
        & {}
        & {}
        & {}
        \\
          1
        & \ddots
        & {}
        & {}
        \\
          {}
        & \ddots
        & \ddots
        & {}
        \\
          {}
        & {}
        & 1
        & \lambda
      \end{bsmallmatrix}
    }
  \end{tikzcd}
\]
with~$\lambda \in k$ and~$n \geq 1$ form a set of representatives for the isomorphism classes of {\fd} indecomposable representations of~$Q$ over~$k$.





\subsection{}

For the matrix
\[
  A
  \defined
  \begin{bmatrix*}[r]
    0 & -1  \\
    1 &  0
  \end{bmatrix*}
  \in
  \mat{2}{\Real}
\]
the only~\dash{$A$}{invariant} subspaces of~$\Real^2$ are~$0$ and~$\Real^2$, because the vectors~$v$ and~$Av$ are linearly independent for every~$v \in V$,~$v \neq 0$.
The representation
\[
  \begin{tikzcd}
    \Real^2
    \arrow[loop right]{}[right]{A}
  \end{tikzcd}
\]
is therefore irreducible, and hence indecomposable.

But when we change the base field from~$\Real$ to~$\Complex$ then the matrix~$A$ becomes diagonalizable with
\[
    \Complex^2
  = \gen{ e_1 + i e_2 }_\Complex
    \oplus
    \gen{ e_1 - i e_2 }_\Complex
\]
being a decomposition into nonzero~\dash{$A$}{invariant} subspaces.
The representation
\[
  \begin{tikzcd}
    \Complex^2
    \arrow[loop right]{}[right]{A}
  \end{tikzcd}
\]
is therefore decomposable.


