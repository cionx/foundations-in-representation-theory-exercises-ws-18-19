\section{}





\subsection{}

For~$f \in \Hom_A(M,N)$ and~$z \in P$ consider the map
\[
          \tilde{\Phi}(f,z)
  \colon  M
  \to     N \tensor_B P \,,
  \quad   x
  \mapsto f(x) \tensor z \,.
\]
The map~$\tilde{\Phi}(f,z)$ is a homomorphism of left~{\modules{$A$}} because
\begin{gather*}
  \begin{aligned}
      \tilde{\Phi}(f,z)(x_1 + x_2)
  &=  f(x_1 + x_2) \tensor z
   =  ( f(x_1) + f(x_2) ) \tensor z \\
  &=  f(x_1) \tensor z + f(x_2) \tensor z
   =  \tilde{\Phi}(f,z)(x_1) + \tilde{\Phi}(f,z)(x_2)
  \end{aligned}
\shortintertext{and}
    \tilde{\Phi}(f,z)(ax)
  = f(ax) \tensor z
  = (af(x)) \tensor z
  = a ( f(x) \tensor z )
  = a \tilde{\Phi}(f,z)(x)
\end{gather*}
for all~$x, x_1, x_2 \in M$,~$a \in A$.
This shows that~$\tilde{\Phi}(f,z)$ is a {\welldef} element of~$\Hom_A(M, N \tensor_B P)$, hence that the map
\[
          \tilde{\Phi}
  \colon  \Hom_A(M,N) \times P
  \to     \Hom_A(M, N \tensor_B P)
\]
is {\welldef}.
The map~$\tilde{\Phi}$ is~{\balanced{$B$}} because
\begin{gather*}
  \begin{aligned}
        \tilde{\Phi}(f_1 + f_2, z)(x)
    &=  (f_1 + f_2)(x) \tensor z
     =  (f_1(x) + f_2(x)) \tensor z \\
    &=  f_1(x) \tensor z + f_2(x) \tensor z
     =  \tilde{\Phi}(f_1,z)(x) + \tilde{\Phi}(f_2,z)(x) \\
    &=  ( \tilde{\Phi}(f_1,z) + \tilde{\Phi}(f_2,z) )(x) \,,
  \end{aligned}
  \\
      \tilde{\Phi}(f, z_1 + z_2)(x)
    = f(x) \tensor (z_1 + z_2)
    = f(x) \tensor z_1 + f(x) \tensor z_2
    = \tilde{\Phi}(f, z_1) + \tilde{\Phi}(f, z_2) \,,
  \\
      \tilde{\Phi}(\lambda f,z)(x)
    = (\lambda f)(x) \tensor z
    = (\lambda f(x)) \tensor z
    = \lambda (f(x) \tensor z)
    = \lambda \tilde{\Phi}(f,z)(x)
    = ( \lambda \tilde{\Phi}(f,z) )(x)
  \\
      \tilde{\Phi}(f, \lambda z)(x)
    = f(x) \tensor (\lambda z)
    = \lambda (f(x) \tensor z)
    = \lambda \tilde{\Phi}(f,z)(x)
    = ( \lambda \tilde{\Phi}(f,z) )(x)
  \\
      \tilde{\Phi}(fb,z)(x)
    = (fb)(x) \tensor z
    = (f(x)b) \tensor z
    = f(x) \tensor (bz)
    = \tilde{\Phi}(f,bz)(x)
\end{gather*}
for all~$f, f_1, f_2 \in \Hom_A(M,N)$~$x, x_1, x_2 \in M$,~$\lambda \in k$,~$z \in P$.
It follows that~$\tilde{\Phi}$ induces a {\welldef}~{\klin} map
\begin{align*}
            \Phi
   \colon   \Hom_A(M, N) \tensor_B P
  &\to      \Hom_A(M, N \tensor_B P) \,,
  \\
            f \tensor z
  &\mapsto  \left[
                      \tilde{\Phi}(f,z)
              \colon  x
              \mapsto f(x) \tensor z
            \right] \,,
\end{align*}
as desired.





\subsection{}

We consider the commutative~{\kalgs}~$A, B \defined k[t]/(t^2)$ and the~{\modules{$A$}}~$N \defined A$ and~$M, P \defined A/(t) \cong k[t]/(t)$.
The~{\module{$A$}}~$\Hom_A(M,N) = \Hom_A(M,A) = \dual{M}$ is the dual module of~$M$, and it holds that~$N \tensor_B P \cong P = M$.%
\footnote{Here we implicitely use that over commutatives rings the \enquote{\dash{non}{commutative}} tensor product coincides with the \enquote{commutative} one.}
We thus have to show that the homomorphism
\[
          \Phi
  \colon  \dual{M} \tensor_A M
  \to     \End_A(M) \,,
  \quad   f \tensor x
  \mapsto (z \mapsto f(z) x)
\]
is neither surjective nor injective.
We do so by showing that~$\dual{M} \tensor_A M \cong M$ and~$\End_A(M) \cong M$, but that the homomorphism~$M \to M$ corresponding to~$\Phi$ is the zero homomorphism.

We first note that
\[
        \dual{M}
  =     \Hom_A(M, A)
  =     \Hom_A( A/(t), A)
  =     \{ \text{\dash{$t$}{torsion} of~$A$} \}
  =     tA
  \cong A/(t)
  =     M \,,
\]
where the inverse isomorphism~$M \to \dual{M}$ is given on representatives given by
\[
                  M
  \xlongto{\sim}  \dual{M} \,,
  \quad           [p]
  \mapsto         ( [q] \mapsto qtp = tpq ) \,.
\]
We also have an isomorphism
\[
        M \tensor_A M
  =     A/(t) \tensor_A A/(t)
  \cong A/( (t) + (t) )
  =     A/(t)
  =     M \,,
\]
whose inverse is given by
\[
                  M
  \xlongto{\sim}  M \tensor_A M \,,
  \quad           [p]
  \mapsto         [p] \tensor [1]
  =               [1] \tensor [p] \,.
\]
Lastly, we have an isomorphism
\[
        \End_A(M)
  =     \End_A(A/(t))
  \cong \End_{A/(t)}(A/(t))
  \cong A/(t)
  =     M \,,
\]
which is given by
\[
                  \End_A(M)
  \xlongto{\sim}  M \,,
  \quad           f
  \mapsto         f(1) \,.
\]
The overall resulting composition
\[
                  M
  \xlongto{\sim}  M \tensor_A M
  \xlongto{\sim}  \dual{M} \tensor_A M
  \xlongto{\Phi}  \End_A(M)
  \xlongto{\sim}  M
\]
is on elements given by
\[
          [p]
  \mapsto [p] \tensor [1]
  \mapsto ( [q] \mapsto tpq ) \tensor [1]
  \mapsto ( [q] \mapsto tpq [1] = [tpq] )
  \mapsto [tp]
  =       t [p] \,,
\]
and thus by multiplication with~$t$.
It follows that this composition is the zero map because~$t M = 0$.
We thus have the following commutative diagram in which all vertical arrows are isomorphisms:
\[
  \begin{tikzcd}
      \dual{M} \tensor_A M
      \arrow{r}[above]{\Phi}
    & \End_A(M)
      \arrow{dd}[right]{\sim}
    \\
      M \tensor M
      \arrow{u}[left]{\sim}
    & {}
    \\
      M
      \arrow{u}[left]{\sim}
      \arrow{r}[above]{0}
    & M
  \end{tikzcd}
\]
Because the zero endomorphism~$M \to M$ is neither injective nor surjective, the same follows for~$\Phi$.

















\subsection{}



\subsubsection{If~$M$ is Free}

Suppose that~$M$ is free of rank~$n$ with basis~$x_1, \dotsc, x_n$.
Let~$\varphi \colon A^n \to M$ be the unique isomorphism of left~{\modules{$A$}} with~$\varphi(e_i) = x_i$ for all~$i = 1, \dotsc, n$.
Then
\begin{align}
       {}&  \Hom_A(M, N) \tensor_B P  \notag  \\
  \cong{}&  \Hom_A(A^{\oplus n}, N) \tensor_B P \label{iso first} \\
  \cong{}&  \Hom_A(A, N)^{\times n} \tensor_B P \notag  \\
  \cong{}&  N^{\times n} \tensor_B P  \notag  \\
        =&  N^{\oplus n} \tensor_B P  \notag  \\
  \cong{}&  (N \tensor_B P)^{\oplus n}  \notag
\end{align}
and also
\begin{align}
       {}&  \Hom_A(M, N \tensor_B P)  \notag  \\
  \cong{}&  \Hom_A(A^{\oplus n}, N \tensor_B P) \label{iso second}  \\
  \cong{}&  \Hom_A(A, N \tensor_B P)^{\times n} \notag  \\
  \cong{}&  (N \tensor_B P)^{\times n}  \notag  \\
      ={}&  (N \tensor_B P)^{\oplus n}  \notag  \,,
\end{align}
where the isomorphisms~\eqref{iso first} and~\eqref{iso second} are induced by~$\varphi$.
The first of the above two isomorphisms is altogether given by
\begin{align*}
            \Hom_A(M, N) \tensor P
  &\to      (N \tensor_B P)^{\oplus n} \,,
  \\
            f \tensor z
  &\mapsto  ( f(x_1) \tensor z, \dotsc, f(x_n) \tensor z ) \,,
\end{align*}
and the second isomorphism is altogether given by
\begin{align*}
            \Hom_A(M, N \tensor_B P)
  &\to      (N \tensor P)^{\oplus n} \,,
  \\
            g
  &\mapsto  ( g(x_1), \dotsc, g(x_n) ) \,.
\end{align*}
The diagram
\[
  \begin{tikzcd}
      \Hom_A(M, N) \tensor_B P
      \arrow{dr}[below left]{\sim}
      \arrow{rr}[above]{\Phi}
    & {}
    & \Hom_A(M, N \tensor_B P)
      \arrow{dl}[below right]{\sim}
    \\
      {}
    & (N \tensor_B P)^{\oplus n}
    & {}
  \end{tikzcd}
\]
commutes, and so it follows that~$\Phi$ is an isomorphism.





\subsubsection{If~$P$ is Free}

Suppose that~$P$ is free of rank~$n$ and with basis~$z_1, \dotsc, z_n$.
Let~$\varphi \colon B^n \to P$ be the unique isomorphism of left~{\modules{$B$}} with~$\varphi(z_i) = e_i$ for every~$i = 1 \dotsc, n$.
Then
\begin{align}
       {}&  \Hom_A(M, N) \tensor_B P  \notag  \\
  \cong{}&  \Hom_A(M, N) \tensor_B B^{\oplus n} \label{third iso} \\
  \cong{}&  \left( \Hom_A(M, N) \tensor_B B \right)^{\oplus n}  \notag  \\
  \cong{}&  \Hom_A(M, N)^{\oplus n} \notag
\end{align}
and also
\begin{align}
       {}&  \Hom_A(M, N \tensor_B P)  \notag  \\
  \cong{}&  \Hom_A(M, N \tensor_B B^{\oplus n}) \label{fourth iso}  \\
  \cong{}&  \Hom_A(M, N^{\oplus n}) \notag  \\
  \cong{}&  \Hom_A(M, N^{\times n}) \notag  \\
  \cong{}&  \Hom_A(M, N)^{\times n}  \notag  \\
  \cong{}&  \Hom_A(M, N)^{\oplus n} \notag  \,,
\end{align}
where the isomorphisms~\eqref{third iso} and~\eqref{fourth iso} are induced by~$\varphi$.
The inverse of the first isomorphism is given by
\begin{align*}
            \Hom_A(M,N)^{\oplus n}
  &\to      \Hom_A(M,N) \tensor_B P \,,
  \\
            (f_1, \dotsc, f_n)
  &\mapsto  f_1 \tensor z_1 + \dotsb + f_n \tensor z_n \,,
\end{align*}
and the inverse of the second isomorphism is given by
\begin{align*}
              \Hom_A(M,N)^{\oplus n}
  \to{}&      \Hom_A(M, N \tensor_B P) \,,
  \\
              (f_1, \dotsc, f_n)
  \mapsto{}&  [x \mapsto f_1(x) \tensor z_1 + \dotsb + f_n(x) \tensor z_n]
  \\
              {}
  &=          \Phi( f_1 \tensor z_1 + \dotsb + f_n \tensor z_n ) \,.
\end{align*}
This shows that the diagram
\[
  \begin{tikzcd}
      \Hom_A(M, N) \tensor_B P
      \arrow{rr}[above]{\Phi}
    & {}
    & \Hom_A(M, N \tensor_B P)
    \\
      {}
    & \Hom_A(M, N)^{\oplus n}
      \arrow{ul}[below left]{\sim}
      \arrow{ur}[below right]{\sim}
    & {}
  \end{tikzcd}
\]
commutes, hence that~$\Phi$ is an isomorphism.




