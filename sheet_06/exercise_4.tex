\section{}

\begin{lemma}
  \label{epi in component}
  Let~$X, Y_1, Y_2 \in \Ob(\Ccat)$ be objects, let~$f_1 \colon X \colon Y_1$ be an epimorphism and let~$f_2 \colon X \to Y_2$ be a morphism.
  Then the morphism~$f \colon X \xto{\begin{bsmallmatrix} f_1 & f_2 \end{bsmallmatrix}} Y_1 \oplus Y_2$ is also an epimorphism.
\end{lemma}

\begin{proof}
  Let~$g_1, g_2 \colon Y_1 \oplus Y_2 \to Z$ be two parallel morphism with~$g_1 \circ f = g_2 \circ f$.
  If~$i \colon Y_1 \to Y_1 \oplus Y_2$ is the canonical morphism into the second summand then
  \[
              g_1 \circ f = g_2 \circ f
    \implies  g_1 \circ f \circ i = g_2 \circ f \circ i
    \implies  g_1 \circ f_1 = g_2 \circ f_1
    \implies  g_1 = g_2
  \]
  because~$f_1$ is an epimorphism.
\end{proof}

\begin{lemma}
  \label{epi is cokernel of kernel}
  Let~$f \colon X \to Y$ be an epimorphism in an abelian category.
  Then~$f$ is a cokernel of its kernel.
\end{lemma}

\begin{proof}
  The zero morphism~$Y \to \coker(f)$ is a cokornel of~$f$ because~$f$ is an epimorphism, and the identity~$\id_Y \colon Y \to Y$ is a therefore an image of~$f$.
  Together with the canonical factorization~$\coim(f) \to \im(f) = Y$ of the morphism~$f$, which is an isomorphism beause~$\Acat$ is abelian, we get the following commutative triangle:
  \[
    \begin{tikzcd}
        X
        \arrow{r}[above]{f}
        \arrow{d}
      & Y
      \\
        \coim(f)
        \arrow{ur}[below right]{\sim}
      & {}
    \end{tikzcd}
  \]
  As the coimage~$X \to \coim(f)$ is a cokernel of~$\ker(f) \to X$, this diagram shows that~$f$ is a cokernel of~$\ker(f) \to X$.
\end{proof}

It follows from \cref{characterization of pullbacks} that in the sequence
\[
    X'
  \xlongto{ \begin{bsmallmatrix} g' \\ f' \end{bsmallmatrix} }
    X \oplus Y'
  \xlongto{ \begin{bsmallmatrix} f & -g \end{bsmallmatrix} }
    Y
\]
the morphism~$X' \to X \oplus Y'$ is a kernel of the morphism~$X \oplus Y' \to Y$, and it follows from \cref{epi in component} that the morphism~$X \oplus Y' \to Y$ is an epimorphism.
It follows from \cref{epi is cokernel of kernel} that the morphism~$X \oplus Y' \to Y$ is a cokernel of the morphism~$X' \to X \oplus Y'$.
The diagram given diagram is therefore a pushout square by \cref{pushout via exact sequence}.
It hence follows from \cref{pushouts and epis} that~$f'$ is again an epimorphism (beause~$\Acat$ is abelian).
