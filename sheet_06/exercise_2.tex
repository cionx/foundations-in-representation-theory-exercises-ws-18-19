\section{}


If~$h \colon Z \to X$ is any morphism in~$\Ccat$, then for~$\alpha \defined g' \circ h$ and~$\beta \defined f' \circ h$ the following diagram commutes:
\[
  \begin{tikzcd}
      Z
      \arrow[dashed,bend left]{drr}[above right]{\alpha}
      \arrow{dr}[above right]{h}
      \arrow[dashed,bend right]{ddr}[below left]{\beta}
    & {}
    & {}
    \\
      {}
    & X'
      \arrow{r}[above]{g'}
      \arrow{d}[right]{f'}
    & X
      \arrow{d}[right]{f}
    \\
      {}
    & Y'
      \arrow{r}[above]{g}
    & Y
  \end{tikzcd}
\]
The morphism~$h$ is uniquely determined by the compositions~$\alpha$ and~$\beta$ because the given diagram is a {\pb} square.
The morphisms~$\alpha$ is uniquely determined by the composition~$f \circ \alpha$ because~$f$ is a monomorphism, and this composition is given by~$f \circ \alpha = g \circ \beta$.
Hence~$h$ is uniquely determined by~$\beta = f' \circ h$, which shows that~$f'$ is a monomorphism.


\begin{remark}
  In an abelian category~$\Acat$ the converse also holds, i.e.\ if the diagram
  \[
    \begin{tikzcd}
        X'
        \arrow{r}[above]{g'}
        \arrow{d}[right]{f'}
      & X
        \arrow{d}[right]{f}
      \\
        Y'
        \arrow{r}[above]{g}
      & Y
    \end{tikzcd}
  \]
  in~$\Acat$ is a {\pb} square and~$f'$ is a monomorphism then~$f$ is also a monomorphism.
  Indeed, the canonical morphism~$\ker(f) \to X$ fits into the following commutative diagram:
  \[
    \begin{tikzcd}
        \ker(f)
        \arrow[dashed,bend left]{drr}
        \arrow[dashed,bend right]{ddr}[below left]{0}
      & {}
      & {}
    \\
        {}
      & X'
        \arrow{r}[above]{g'}
        \arrow{d}[right]{f'}
      & X
        \arrow{d}[right]{f}
      \\
        {}
      & Y'
        \arrow{r}[above]{g}
      & Y
    \end{tikzcd}
  \]
  It follows that there exist a unique morphism~$\lambda \colon \ker(f) \to X'$ which makes the diagram
  \[
    \begin{tikzcd}
        \ker(f)
        \arrow[bend left]{drr}
        \arrow[dashed]{dr}[above right]{\lambda}
        \arrow[bend right]{ddr}[below left]{0}
      & {}
      & {}
    \\
        {}
      & X'
        \arrow{r}[above]{g'}
        \arrow{d}[right]{f'}
      & X
        \arrow{d}[right]{f}
      \\
        {}
      & Y'
        \arrow{r}[above]{g}
      & Y
    \end{tikzcd}
  \]
  commute.
  It follows from~$0 = f' \circ \lambda$ and~$f$ being a monomorphism that also~$\lambda = 0$.
  The canonical morphism~$\ker(f) \to X$ is therefore given by
  \[
      g' \circ \lambda
    = g' \circ 0
    = 0 \,.
  \]
  This shows that~$\ker(f) = 0$ and hence that~$f$ is a monomorphism.
\end{remark}


\begin{remark}
  \label{pushouts and epis}
  It holds dually for a pushout square
  \[
    \begin{tikzcd}
        X
        \arrow{r}[above]{g}
        \arrow{d}[left]{f}
      & X'
        \arrow{d}[left]{f'}
      \\
        Y
        \arrow{r}[above]{g'}
      & Y'
    \end{tikzcd}
  \]
  that if~$f$ is an epimorphism, then~$f'$ is also an epimorphism;
  in an abelian category, the converse also holds.
\end{remark}



