\section{}

\begin{proposition}
  \label{characterization of pullbacks}
  Let~$\Acat$ be an additive category.
  Then a diagram
  \[
    \begin{tikzcd}
        X'
        \arrow{r}[above]{g'}
        \arrow{d}[right]{f'}
      & X
        \arrow{d}[right]{f}
      \\
        Y'
        \arrow{r}[above]{g}
      & Y
    \end{tikzcd}
  \]
  in~$\Acat$ is a {\pb} square if and only if in the squence
  \begin{equation}
    \label{left exact sequence}
      X'
    \xlongto{ \begin{bsmallmatrix} g' \\ f' \end{bsmallmatrix} }
      X \oplus Y'
    \xlongto{ \begin{bsmallmatrix} f & -g \end{bsmallmatrix} }
      Y \,,
  \end{equation}
  the morphism~$X' \to X \oplus Y'$ is a kernel of the morphism~$X \oplus Y' \to Y$ (i.e.\ the sequence is left exact).
\end{proposition} 

\begin{proof}
  Let~$p \colon X \oplus Y' \to X$ and~$q \colon X \oplus Y' \to Y'$ be the canonical projections belonging to the biproduct~$X \oplus Y$ and let
  \[
      d
    \colon
      X \oplus Y'
    \xlongto{\begin{bsmallmatrix} f & -g \end{bsmallmatrix}}
      Y \,.
  \]
  It then holds for every object~$Z \in \Ob(\Ccat)$ and every two morphisms~$\alpha \colon Z \to X$ and~$\beta \colon Z \to Y$ that
  \[
          f \circ \alpha = g \circ \beta
    \iff  f \circ \alpha - g \circ \beta = 0
    \iff  \begin{bmatrix}
            f & -g
          \end{bmatrix}
          \begin{bmatrix}
            \alpha  \\
            \beta
          \end{bmatrix}
          = 0
    \iff  d
          \circ
          \begin{bmatrix}
            \alpha  \\
            \beta
          \end{bmatrix}
          = 0
  \]
  Let~$k \colon X' \to X \oplus Y'$ be a morphism, which is uniquely of the form
  \[
      k
    = \begin{bmatrix}
        g'  \\
        f'
      \end{bmatrix}
  \]
  for some morphisms~$q' \colon X' \to X$ and~$f' \colon X' \to Y'$.
  We find from the above calculation that~$d \circ k = 0$ if and only if the square
  \[
    \begin{tikzcd}
        X'
        \arrow{r}[above]{g'}
        \arrow{d}[right]{f'}
      & X
        \arrow{d}[right]{f}
      \\
        Y'
        \arrow{r}[above]{g}
      & Y
    \end{tikzcd}
  \]
  commutes.
  We have moreover find that
  \begin{align*}
        {}& \begin{tabular}{c}
              there exist for all morphisms
              $\alpha \colon Z \to X$ and~$\beta \colon Z \to Y'$ with~$f \circ \alpha = g \circ \beta$ \\
              a unique morphism~$\lambda \colon Z \to X'$ with~$g' \circ \lambda = \alpha$ and~$f' \circ \lambda = \beta$
            \end{tabular} \\
    \iff{}& \begin{tabular}{c}
              there exist for all morphisms
              $\alpha \colon Z \to X$ and~$\beta \colon Z \to Y'$ with~$d \circ \begin{bsmallmatrix} \alpha \\ \beta \end{bsmallmatrix} = 0$ \\
              a unique morphism~$\lambda \colon Z \to X'$ with~$k \circ \lambda = \begin{bsmallmatrix} \alpha \\ \beta \end{bsmallmatrix}$
            \end{tabular} \\
    \iff&{} \begin{tabular}{c}
              there exist for every morphism
              $\gamma \colon Z \to X \oplus Y'$ with~$d \circ \gamma = 0$ \\
              a unique morphism~$\lambda \colon Z \to X'$ with~$k \circ \lambda = \gamma$ \,.
            \end{tabular}
  \end{align*}
  This shows that~$(X', f', g')$ is a {\pb} of the given morphisms~$X \xto{f} Y \xfrom{g} Y'$ if and only if the corresponding morphism~$k \colon Z \to X \oplus Y$ is a kernel of~$d$.
\end{proof}

\begin{remark}
  \label{variation for pullback}
  Instead of the sequence~\eqref{left exact sequence} one can also use variations such as
  \begin{gather*}
      X'
    \xlongto{ \begin{bsmallmatrix} g' \\ f' \end{bsmallmatrix} }
      X \oplus Y'
    \xlongto{ \begin{bsmallmatrix} -f & g \end{bsmallmatrix} }
      Y \,,
  \\
      X'
    \xlongto{ \begin{bsmallmatrix} g' \\ -f' \end{bsmallmatrix} }
      X \oplus Y'
    \xlongto{ \begin{bsmallmatrix} f & g \end{bsmallmatrix} }
      Y \,,
  \\
      X'
    \xlongto{ \begin{bsmallmatrix} - g' \\ f' \end{bsmallmatrix} }
      X \oplus Y'
    \xlongto{ \begin{bsmallmatrix} f & g \end{bsmallmatrix} }
      Y \,.
  \end{gather*}
\end{remark}


\begin{remark}
  \label{pushout via exact sequence}
  The dual version of \cref{characterization of pullbacks} states that in an additive category~$\Acat$, a diagram
  \[
    \begin{tikzcd}
        X
        \arrow{r}[above]{f}
        \arrow{d}[right]{g}
      & Y
        \arrow{d}[right]{g'}
      \\
        X'
        \arrow{r}[above]{f'}
      & Y'
    \end{tikzcd}
  \]
  is a pushout square if and only if in the sequence
  \[
      X
    \xlongto{ \begin{bsmallmatrix} f' \\ g \end{bsmallmatrix} }
      X' \oplus Y
    \xlongto{ \begin{bsmallmatrix} f' & -g \end{bsmallmatrix} }
      Y'
  \]
  the morphism~$X' \oplus Y \to Y'$ is a cokernel of the morphism~$X \to X' \oplus Y$ (i.e.\ the sequence is right exact).
  One can again vary this sequence, as done in \cref{variation for pullback} for pullbacks.
\end{remark}


% Let~$p \colon X \times Y' \to X$ and~$q \colon X \times Y' \to Y'$ be the canonical projections belonging to the biproduct~$X \otimes Y'$
% Let~$k \colon X' \to X \times Y'$ be a kernel of the morphism
% \[
%             d
%   \defined  f \circ p - g \circ q
%   \colon    X \times Y'
%   \to       Y \,.
% \]
% For~$f' \defined q \circ k$ and~$g' \defined q \circ k$ we then have the following diagram, in which the upper and left triangles commute:
% \[
%   \begin{tikzcd}
%       X'
%       \arrow{rr}[above]{g'}
%       \arrow{dr}[below left]{k}
%       \arrow{dd}[right]{f'}
%     & {}
%     & X
%       \arrow{dd}[right]{f}
%     \\
%       {}
%     & X \times Y'
%       \arrow{ur}[below right]{p}
%       \arrow{dr}[above right]{d}
%       \arrow{dl}[below right]{q}
%     & {}
%     \\
%       Y'
%       \arrow{rr}[above]{g}
%     & {}
%     & Y
%   \end{tikzcd}
% \]
% The outer square also commutes because
% \[
%     0
%   = d \circ k
%   = (f \circ p - g \circ c) \circ k
%   = f \circ p \circ k - g \circ q \circ k
%   = f \circ g' - g \circ f'
% \]
% and hence~$f \circ g' - g \circ f'$.
% 
% If more generally~$Z \in \Ob(\Ccat)$ is an object and~$\alpha \colon Z \to X$ and~$\beta \colon Z \to Y'$ are morphisms, with~$f \circ \alpha = g \circ \beta$ then it holds for the morphism~$(\alpha,\beta) \colon Z \to X \times Y$ that
% \begin{align*}
%       d \circ (\alpha,\beta)
%   &=  (f \circ p - g \circ q) \circ (\alpha,\beta)  \\
%   &=  f \circ p \circ (\alpha,\beta) - g \circ q \circ (\alpha,\beta)
%    =  f \circ \alpha - g \circ \beta
%    =  0 \,.
% \end{align*}
% Hence~$(\alpha, \beta)$ factors uniquely through the kernel~$k \colon X' \to X \times Y'$ of~$d$, i.e.\ there exists a unique morphism~$\gamma \colon Z \to X'$ with~$k \circ \gamma = (\alpha, \beta)$.
% \[
%   \begin{tikzcd}[column sep={6em,between origins}]
%       Z
%       \arrow[bend left]{drrr}[above right]{\alpha}
%       \arrow[bend right]{dddr}[below left]{\beta}
%       \arrow[dashed]{dr}[below left]{\gamma}
%     & {}
%     & {}
%     & {}
%     \\
%       {}
%     & X'
%       \arrow{rr}[above]{g'}
%       \arrow{dr}[below left]{k}
%       \arrow{dd}[right]{f'}
%     & {}
%     & X
%       \arrow{dd}[right]{f}
%     \\
%       {}
%     & {}
%     & X \times Y'
%       \arrow[dashed, from=uull, bend left, crossing over, above right, "{(\alpha,\beta)}"]
%       \arrow{ur}[below right]{p}
%       \arrow{dr}[above right]{d}
%       \arrow{dl}[below right]{q}
%     & {}
%     \\
%       {}
%     & Y'
%       \arrow{rr}[above]{g}
%     & {}
%     & Y
%   \end{tikzcd}
% \]
% It then holds that
% \begin{gather*}
%     g' \circ \gamma
%   = p \circ k \circ \gamma
%   = p \circ (\alpha, \beta)
%   = \alpha
% \intertext{and}
%     f' \circ \gamma
%   = q \circ k \circ \gamma
%   = q \circ (\alpha, \beta)
%   = \beta \,.
% \end{gather*}
% 
% Suppose on the other hand that~$\gamma' \colon Z \to X'$ is another morphism which makes the diagram
% \[
%   \begin{tikzcd}
%       Z
%       \arrow[bend left]{drr}[above right]{\alpha}
%       \arrow[bend right]{ddr}[below left]{\beta}
%       \arrow{dr}[above right]{\gamma}
%     & {}
%     & {}
%     \\
%       {}
%     & X'
%       \arrow{r}[above]{g'}
%       \arrow{d}[right]{f'}
%     & X
%       \arrow{d}[right]{f}
%     \\
%       {}
%     & Y'
%       \arrow{r}[above]{g}
%     & Y
%   \end{tikzcd}
% \]
% commute.
% Then the composition~$k \circ \gamma'$ satisfies
% \begin{gather*}
%     p \circ k \circ \gamma'
%   = g' \circ \gamma'
%   = \alpha
% \shortintertext{and}
%     q \circ k \circ \gamma'
%   = f' \circ \gamma'
%   = \beta \,,
% \end{gather*}
% and is hence given by~$k \circ \gamma' = (\alpha, \beta)$.
% It follows that~$\gamma = \gamma'$ by the uniqueness of~$\gamma$.
% 
% 


