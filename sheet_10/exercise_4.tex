\section{}
\label{projective iff summand of free}





\subsection{}

This follows from part~\ref{characterization of projective modules} because~$A$ is free as an~{\module{$A$}}.





\subsection{}
\label{characterization of projective modules}

Suppose first that~$F$ is a free~{\module{$A$}}.
Let~$f \colon M \to N$ is an epimorphism of~{\modules{$A$}} and let~$g \colon F \to N$ be any homomorphism of~{\modules{$A$}}.
There exists by assumption a basis~$(b_i)_{i \in I}$ of~$F$, and for every~$i \in I$ an element~$m_i \in M$ with~$f(m_i) = g(b_i)$.
It follows for the unique homomorphism of~{\modules{$A$}}~$g' \colon F \to M$ with~$g'(b_i) = m_i$ for every~$i \in I$ that
\[
    f(g'(b_i))
  = f(m_i)
  = g(m_i)
\]
for every~$i \in I$, and hence that~$f \circ g' = g$.
This means that the triangle
\[
  \begin{tikzcd}
      {}
    & P
      \arrow{d}[right]{g}
      \arrow[dashed]{dl}[above left]{g'}
    \\
      M
      \arrow{r}[below]{f}
    & N
  \end{tikzcd}
\]
commutes.
This shows that~$F$ is projective, and hence that every free module is projective.

It follows that direct summands of free modules are again projective, because direct summands of projective modules are again projective.
(As known from the lecture and shown in Exercise~2.)

Suppose on the other hand that~$P$ is a projective~{\module{$A$}}.
Then there exists for the free~{\module{$A$}~$F$ with basis~$(b_p)_{p \in P}$ a unique homomorphism of~{\modules{$A$}}~$f \colon F \to P$ with~$f(b_p) = p$ for every~$p \in P$.
The homomorphism~$f$ is an epimorphism and the resulting short exact sequence
\[
  0
  \to
  \ker(f)
  \to
  F
  \to
  P
  \to
  0
\]
splits because~$P$ is projective.
Hence~$P \oplus \ker(f) \cong F$ is free, which shows that~$P$ is a direct summand of a free module.




