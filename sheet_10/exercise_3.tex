\section{}





\subsection{}

For an object~$X \in \Ob(\Acat)$ and any bounded chain complex~$\Ccc \in \Ob(\Chh_{\geq 0}(\Acat))$, a morphism of chain complexes~$f \colon \Ccc \to I_0(X)$ corresponds to the choice of a morphism~$f_0 \colon C_0 \to X$ that makes the diagram
\[
  \begin{tikzcd}
      {}
      \arrow{r}
    & C_2
      \arrow{r}[above]{d_2}
      \arrow[dashed]{d}
    & C_1
      \arrow{r}[above]{d_1}
      \arrow[dashed]{d}
    & C_0
      \arrow[dashed]{d}[right]{f_0}
    \\
      {}
      \arrow{r}
    & 0
      \arrow{r}
    & 0
      \arrow{r}
    & X
  \end{tikzcd}
\]
commute.
The commutativity of this diagram means that~$f_0 d_1 = 0$.
Such morphisms are in~{\onetoone} correspondence to morphism~$\induced{f_0} \colon \coker(d_1) \to X$ through the universal property of the cokernel.
The morphisms~$f_0$ and~$\induced{f_0}$ are more specifically related by the commutativity of the following triangle:
\[
  \begin{tikzcd}
      C_0
      \arrow{r}[above]{f_0}
      \arrow{d}[left]{p_{\Ccc}}
    & X
      \\
      \coker(d_1)
      \arrow[dashed]{ur}[below right]{\induced{f_0}}
    & {}
  \end{tikzcd}
\]
We hence define the functor~$L \colon \Chh_{\geq 0}(\Acat)$ on objects by
\[
  L(\Ccc)
  =
  \coker(d^{\,C}_1) \,.
\]
For every morphism of chain complexes~$f \colon \Ccc \to \Dcc$ we let
\[
  L(f)
  \colon
  \coker(d^{\,C}_1)
  \to
  \coker(d^{\,C}_1)
\]
be the unique morphism in~$\Acat$ that makes the square
\[
  \begin{tikzcd}
      C_0
      \arrow{r}[above]{p_{\Ccc}}
      \arrow{d}[left]{f_0}
    & L(\Ccc)
      \arrow[dashed]{d}[right]{L(f)}
    \\
      D_0
      \arrow{r}[below]{p_{\Dcc}}
    & L(\Dcc)
  \end{tikzcd}
\]
commute.
(That this defines a functor follows from the functoriality of the cokernel.)

We have argued above as to why the map
\begin{align*}
  \varphi_{\Ccc,X}
  \colon
  \Hom_{\Chh_{\geq 0}(\Acat)}(\Ccc, I_0(X))
  &\to
  \Hom_{\Acat}(L(\Ccc), X)  \,,
  \\
  f
  &\mapsto
  \induced{f_0}
\end{align*}
is a bijection.
This bijection in natural in both~$\Ccc$ and~$X$:
Let~$g \colon \Dcc \to \Ccc$ be a morphism of~\dash{$\Acat$}{valued} bounded chain complexes and let~$h \colon X \to Y$ be a morphism in~$\Acat$.
Then the diagram
\begin{equation}
  \label{equation: naturality square}
  \begin{tikzcd}[column sep = 6em]
      \Hom_{\Chh_{\geq 0}(\Acat)}(\Ccc, I_0(X))
      \arrow{r}[above]{I_0(h) \circ (-) \circ g}
      \arrow{d}[left]{\varphi_{\Ccc,X}}
    & \Hom_{\Chh_{\geq 0}(\Acat)}(\Dcc, I_0(Y))
      \arrow{d}[right]{\varphi_{\Dcc,Y}}
    \\
      \Hom_{\Acat}(L(\Ccc), X)
      \arrow{r}[below]{h \circ (-) \circ L(g)}
    & \Hom_{\Acat}(L(\Dcc), Y)
  \end{tikzcd}
\end{equation}
commutes.
Indeed, it follows for every morphism~$f \in \Hom_{\Chh_{\geq 0}(\Acat)}(\Ccc, I_0(X))$ from the commutativity of the square
\[
  \begin{tikzcd}[column sep = huge]
      \Ccc
      \arrow{r}[above]{f}
    & I_0(X)
      \arrow{d}[right]{I_0(h)}
    \\
      \Dcc
      \arrow{u}[left]{g}
      \arrow{r}[below]{I_0(h) \circ f \circ g}
    & I_0(Y)
  \end{tikzcd}
\]
that in degree~$0$ the following square commutes:
\[
  \begin{tikzcd}[column sep = huge]
      C_0
      \arrow{r}[above]{f_0}
    & X
      \arrow{d}[right]{h}
    \\
      D_0
      \arrow{u}[left]{g_0}
      \arrow{r}[below]{(I(h) \circ f \circ g)_0}
    & Y
  \end{tikzcd}
\]
We may extend this commutative square to the following diagram:
\[
  \begin{tikzcd}[column sep = huge]
      L(\Ccc)
      \arrow{rrr}[above]{\induced{f_0}}
    & {}
    & {}
    & X
      \arrow{ddd}[right]{h}
    \\
      {}
    & C_0
      \arrow{ul}[below left]{p_{\Ccc}}
      \arrow{r}[above]{f_0}
    & X
      \arrow[equal]{ur}
      \arrow{d}[right]{h}
    & {}
    \\
      {}
    & D_0
      \arrow{dl}[above left]{p_{\Dcc}}
      \arrow{u}[left]{g_0}
      \arrow{r}[below]{(I(h) \circ f \circ g)_0}
    & Y
      \arrow[equal]{dr}
    & {}
    \\
      L(\Dcc)
      \arrow{uuu}[left]{L(g)}
      \arrow{rrr}[below]{\induced{(I(h) \circ f \circ g)_0}}
    & {}
    & {}
    & Y
  \end{tikzcd}
\]
The four added trapezoids commute by the constructions of both~$L$ and~$\induced{(-)}$.
It follows that the outer square commutes, because the morphism~$p_{\Dcc}$ is an epimorphsm and
\begin{align*}
      \induced{ (I(h) \circ f \circ g)_0 } \circ p_{\Dcc}
   =  (I(h) \circ f \circ g)_0
  &=  h \circ f_0 \circ g_0 \\
  &=  h \circ \induced{f_0} \circ p_{\Ccc} \circ g_0
   =  h \circ \induced{f_0} \circ L(g) \circ p_{\Dcc}
\end{align*}
This shows that
\[
    \induced{ (I(h) \circ f \circ g)_0 }
  = h \circ \induced{f_0} \circ L(g) \,.
\]
This proves the desired commutativity of the square~\eqref{equation: naturality square}, and whence that the functor~$L$ is left adjoint to the functor~$I_0$.

\begin{remark}
  We have for every bounded chain complex~$\Ccc \in \Chh_{\geq 0}(\Acat)$ that
  \[
    \coker(d_1)
    =
    \Hl_0(\Ccc) \,.
  \]
  The desired left adjoint can therefore also be described as the zeroeth homology~$H_0$.
  But the above explicit description of~$L(\Ccc)$ as~$\coker(d_1)$ works also for unbouded chain complexes, and hence also gives a left adjoint~$\Ch(\Acat) \to \Acat$ of~$I_0 \colon \Acat \to \Ch(\Acat)$.
\end{remark}





\subsection{}

This follows from part~(i) by duality.
The right adjoint~$R \colon \CChh^{\geq 0}(\Acat)$ of the functor~$I^0$ is given on objects by
\[
    R(\Cccc)
  = \ker(d_C^0) \,,
\]
and assigns to every morphism~$f \colon \Cccc \to \Dccc$ of (bounded) cochain complexes the unique morphism~$\ker(d_C^0) \to \ker(d_D^0)$ that makes the square
\[
  \begin{tikzcd}
      \ker(d_C^0)
      \arrow{r}
      \arrow[dashed]{d}[left]{R(f)}
    & C^0
      \arrow{d}[right]{f^0}
    \\
      \ker(d_D^0)
      \arrow{r}
    & D^0
  \end{tikzcd}
\]
commute.



