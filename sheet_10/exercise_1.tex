\section{}





\subsection{}

The statement is \textbf{false}.
As a counterexample we might consider for~$\Acat = \Modl{\Integer}$ the morphism of chain complexes~$f \colon \Ccc \to \Dcc$ that is given by the following commutative diagram:
\[
  \begin{tikzcd}
      \Ccc
      \arrow{d}[right]{f}
    & \dotsb
      \arrow{r}
    & 0
      \arrow{r}
      \arrow{d}
    & 2\Integer
      \arrow{r}
      \arrow{d}
    & \Integer
      \arrow{r}
      \arrow{d}
    & 0
      \arrow{r}
      \arrow{d}
    & \dotsb
    \\
      \Dcc
    & \dotsb
      \arrow{r}
    & 0
      \arrow{r}
    & \Integer
      \arrow{r}
    & \Integer
      \arrow{r}
    & 0
      \arrow{r}
    & \dotsb
  \end{tikzcd}
\]
Here~$2\Integer \to \Integer$ is the inclusion and~$\Integer \to \Integer$ is the identit, and the vertical identity~$\Integer \to \Integer$ is in degree~$0$.
This morphism~$f$ is a monomorphism because it is a monomorphism in each degree.
But the induced morphism~$\Hl_0(f)$ is not a monomorphism because~$\Hl_0(\Ccc) = \Integer/2$ but~$\Hl_0(\Dcc) = 0$.
This shows that~$\Hl_0 \colon \Ch(\Modl{\Integer}) \to \Modl{\Integer}$ does not preserve monomorphisms, and is hence not left exact.





\subsection{}

The statement is \textbf{true}:
We get for the identity morphism~$\id \colon \Ccc \to \Ccc$ the long exact homology sequence of the cone:
\[
  \dotsb
  \to
  \Hl_n(\Ccc)
  \xlongto{\Hl_n(\id)}
  \Hl_n(\Ccc)
  \to
  \Hl_n(\cone(\id))
  \to
  \Hl_{n-1}(\Ccc)
  \to
  \dotsb
\]
The morphism~$\Hl_n(\id) = \id_{\Hl_n(\Ccc)}$ is an isomorphism of every~$n \in \Integer$, hence it follows that~$\Hl_n(\cone(\id)) = 0$ for every~$n \in \Integer$ by the exactness of the above sequence.





\subsection{}

The statement is \textbf{true}:
Every~{\kvs}~$V$ is free, and hence projective by Exercise~4.
We can therefore take the identity~$\id_V \colon V \to V$ to see that the category~$\Modl{k}$ has enough projectives.





\subsection{}

The statement is \textbf{false}:
Consider the monomorphism~$\id \colon \Integer \to \Integer$ and the morphism~$g \colon \Integer \to \Integer$ given by~$g(n) = 2n$.
Then there does not exist an extension~$g' \colon \Integer \to \Integer$ that makes the triangle
\[
  \begin{tikzcd}
      \Integer
      \arrow{r}[above]{g}
      \arrow{d}[left]{\id}
    & \Integer
      \arrow[dashed]{dl}[below right]{g'}
    \\
      \Integer
    & {}
  \end{tikzcd}
\]
commute, because then~$g'(1) \in \Integer$ would need to be a group element with
\[
    2 g'(1)
  = g'(2)
  = g'(g(1))
  = \id(1)
  = 1 \,,
\]
but such an element does not exist.





\addtocounter{subsection}{1}
% TODO: Type this part after the next lecture.





\subsection{}

The statement ist \textbf{false}.
We see in Exercise~4 that an abelian group (i.e.~{\module{$\Integer$}}) is projective if and only if it is a direct summand of a free abelian group.
But free abelian groups are \dash{torsion}{free}, and hence also projective abelian groups are \dash{torsion}{free}, whereas~$\Integer/2$ is a nontrivial torsion group.
This shows that~$\Integer/2$ is not projective.
