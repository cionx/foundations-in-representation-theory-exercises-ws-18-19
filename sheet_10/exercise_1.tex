\section{}





\subsection{}

The statement is \textbf{false}.
As a counterexample we might consider for~$\Acat = \Modl{\Integer}$ the morphism of chain complexes~$f \colon \Ccc \to \Dcc$ that is given by the following commutative diagram:
\[
  \begin{tikzcd}
      \Ccc
      \arrow{d}[right]{f}
    & \dotsb
      \arrow{r}
    & 0
      \arrow{r}
      \arrow{d}
    & 2\Integer
      \arrow{r}
      \arrow{d}
    & \Integer
      \arrow{r}
      \arrow{d}
    & 0
      \arrow{r}
      \arrow{d}
    & \dotsb
    \\
      \Dcc
    & \dotsb
      \arrow{r}
    & 0
      \arrow{r}
    & \Integer
      \arrow{r}
    & \Integer
      \arrow{r}
    & 0
      \arrow{r}
    & \dotsb
  \end{tikzcd}
\]
Here~$2\Integer \to \Integer$ is the inclusion and~$\Integer \to \Integer$ is the identity, and the vertical identity~$\Integer \to \Integer$ is in degree~$0$.
This morphism~$f$ is a monomorphism because it is a monomorphism in each degree.
But the induced morphism~$\Hl_0(f)$ is not a monomorphism because~$\Hl_0(\Ccc) = \Integer/2$ but~$\Hl_0(\Dcc) = 0$.
This shows that~$\Hl_0 \colon \Ch(\Modl{\Integer}) \to \Modl{\Integer}$ does not preserve monomorphisms, and hence is not left exact.





\subsection{}

The statement is \textbf{true}:
We get for the identity morphism~$\id \colon \Ccc \to \Ccc$ the long exact homology sequence of the cone:
\[
  \dotsb
  \to
  \Hl_n(\Ccc)
  \xlongto{\Hl_n(\id)}
  \Hl_n(\Ccc)
  \to
  \Hl_n(\cone(\id))
  \to
  \Hl_{n-1}(\Ccc)
  \to
  \dotsb
\]
The morphism~$\Hl_n(\id) = \id_{\Hl_n(\Ccc)}$ is an isomorphism of every~$n \in \Integer$, hence it follows that~$\Hl_n(\cone(\id)) = 0$ for every~$n \in \Integer$ by the exactness of the above sequence.





\subsection{}

The statement is \textbf{true}:
Every~{\kvs}~$V$ is free, and hence projective by Exercise~4.
We can therefore use the identity~$\id_V \colon V \to V$ to see that the category~$\Modl{k}$ has enough projectives.





\subsection{}

The statement is \textbf{false}:
Consider the morphism~$\id \colon \Integer \to \Integer$ and the monomorphism~$g \colon \Integer \to \Integer$ given by~$g(n) = 2n$.
(This is a monomorphism because it is injective.)
Then there does not exist an extension~$g' \colon \Integer \to \Integer$ of~$g$ that makes the triangle
\[
  \begin{tikzcd}
      \Integer
      \arrow{r}[above]{g}
      \arrow{d}[left]{\id}
    & \Integer
      \arrow[dashed]{dl}[below right]{g'}
    \\
      \Integer
    & {}
  \end{tikzcd}
\]
commute, because~$g'(1) \in \Integer$ would need to be a group element with
\[
    2 g'(1)
  = g'(2)
  = g'(g(1))
  = \id(1)
  = 1 \,.
\]
But such an element does not exist.





\subsection{}

The statement is \textbf{false} in its current form, because~$\Rational$ is not an object in the given category (because~$\Rational$ is not finitely generated as an abelian group).
The original form of the problem, that~$\Rational$ is injective in the category~$\Ab$, is \textbf{true}:

Let~$A$ be an abelian group, let~$g \colon A \to \Rational$ be a homomorphism and let~$f \colon A \to B$ be a monomorphism, i.e.\ an injective group homomorphism.
We need to show that~$g$ extends to a group homomorphism~$g' \colon B \to \Rational$ along~$f$.
For this we may assume w.l.o.g.\ that~$A$ is a subgroup of~$B$ and that~$f \colon A \to B$ is the canonical inclusion.

We find with Zorn’s lemma that there exists a maximal extension~$\tilde{g} \colon \tilde{B} \to \Rational$ of~$g$, i.e.~$A \subseteq \tilde{B} \subseteq B$ is an intermediate group,~$\tilde{g} \colon \tilde{B} \to \Rational$ is an extension of~$g$, and~$\tilde{B}$ is maximal with this property.

Suppose that~$\tilde{B} \neq B$.
Then there exists some~$x \in B$ with~$x \notin \tilde{B}$.
We show for~$B' \defined \tilde{B} + \Integer x$ that the homomorphism~$\tilde{g} \colon \tilde{B} \to \Rational$ can be extended to a homomorphism~$B' \to \Rational$.
We distinguish between two cases:
\begin{itemize}
  \item
    If~$\tilde{B} \cap \Integer x = 0$ then~$B' = \tilde{B} \oplus \Integer x$.
    We can then choose the extension~$g'$ as
    \[
        g'(\tilde{b} + nx)
      = \tilde{g}(\tilde{b})
    \]
    for all~$\tilde{b} \in \tilde{B}$ and~$n \in \Integer$.
  \item
    Otherwise let~$n > 1$ be minimal such that~$nx \in \tilde{B}$.
    (We do not have to consider the case~$n = 1$ because~$x \notin \tilde{B}$.)
    Then~$\tilde{B} \cap \Integer x = \Integer nx$, and we hence have a (right) exact sequence
    \begin{equation}
      \label{right exact}
      \Integer
      \xlongto{\psi}
      \tilde{B} \oplus \Integer
      \xlongto{\varphi}
      B'
      \to
      0 \,,
    \end{equation}
    where~$\varphi(\tilde{b},k) = \tilde{b}+kx$ and~$\varphi(k) = k(nx,-n)$.
    It follows for the homomorphism
    \[
      \tilde{g}'
      \colon
      \tilde{B} \oplus \Integer
      \to
      \Rational \,,
      \quad
      (\tilde{b},k)
      \mapsto
      \tilde{g}(\tilde{b}) + k \frac{\tilde{g}(nx)}{n}
    \]
    that~$\tilde{g}' \circ \psi = 0$ because
    \[
      \tilde{g}'(nx,-n)
      =
      \tilde{g}(nx) - n \frac{\tilde{g}(nx)}{n}
      =
      \tilde{g}(nx) - \tilde{g}(nx)
      =
      0 \,.
    \]
    It follows from the right exactness of~\eqref{right exact} hat~$\tilde{g}'$ factors through a homomorphism~$g' \colon B' \to \Rational$ with
    \[
      g'(\tilde{b} + kx)
      =
      \tilde{g}(\tilde{b}) + k \frac{\tilde{g}(nx)}{n}
    \]
    for all~$\tilde{b} \in \tilde{B}$ and~$k \in \Integer$.
    This is the desired extension of~$\tilde{g}$ to~$B'$.
\end{itemize}

It follows from the maximality of the extension~$(\tilde{B}, \tilde{g})$ that already~$\tilde{B} = B'$, which contradicts~$x \notin \tilde{B}$ (but~$x \in B'$).
We hence find that already~$\tilde{B} = B$, and hence that~$\tilde{g}$ is the desired extension of~$g$ onto~$B$.





\subsection{}

The statement ist \textbf{false}.
We see in Exercise~4 that an abelian group (i.e.~{\module{$\Integer$}}) is projective if and only if it is a direct summand of a free abelian group.
Free abelian groups are \dash{torsion}{free}, and hence projective abelian groups are also \dash{torsion}{free}. But~$\Integer/2$ is a nontrivial torsion group, and hence not projective.



