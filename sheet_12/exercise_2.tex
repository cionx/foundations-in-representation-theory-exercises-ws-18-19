\section{}





\subsection{}

Let~$A$ be an abelian group.
Then there exists for any sufficiently large index set~$I$ a surjective group homomorphism~$p_0 \colon \Integer^{\oplus I} \to A$.
The kernel~$\ker(p_0) \subseteq \Integer^{\oplus I}$ is again free abelian (because subgroups of free abelian groups are again free abelian).
Hence~$\ker(p_0) \cong \Integer^{\oplus J}$ for some suitable index set~$J$.
This gives an exact sequence
\begin{equation}
  \label{short projective resolution for ab}
  \dotsb
  \to
  0
  \to
  0
  \to
  \Integer^{\oplus J}
  \to
  \Integer^{\oplus I}
  \xlongto{p_0}
  A
  \to
  0 \,.
\end{equation}
The abelian groups~$P_0 \defined \Integer^{\oplus I}$ and~$P_1 \defined \Integer^{\oplus J}$ are projective since they are free abelian.
The exact sequence~\eqref{short projective resolution for ab} is therefore a projective resolution for~$A$ as desired.





\subsection{}

For~$n = 0$ the abelian group~$\Integer/n = \Integer$ has the projective resolution
\[
  \underbrace{
  \dotsb
  \to
  0
  \to
  0
  \to
  \Integer
  }_{\Pcc}
  \xlongto{\id}
  \Integer
  \to
  0 \,.
\]
The resulting chain complex
\[
  \Hom_\Integer(\Pcc, \Integer/m)
  =
  \bigl(
    \Hom_\Integer(\Integer,\Integer/m)
    \to
    0
    \to
    0
    \to
    \dotsb
  \bigr)
\]
is isomorphic to the chain complex
\[
  \Integer/m
  \to
  0
  \to
  0
  \to
  \dotsb
\]
By taking the cohomology of this cochain complex we find that
\[
  ( \Right^i \Hom_\Integer(-,\Integer/m) )(\Integer)
  \cong
  \begin{cases}
    \Integer/m  & \text{for~$i = 0$}  \,, \\
    0           & \text{for~$i > 0$}  \,.
  \end{cases}
\]
(One can also see this by using that~$\Right^0 \Hom_\Integer(-,\Integer/m) \cong \Hom_\Integer(-,\Integer/m)$, and also that~$(\Right^i \Hom_\Integer(-,\Integer/m))(\Integer) = 0$ for~$i > 0$ because~$\Integer$ is free abelian and hence projective in~$\Ab$.)

For~$n \neq 0$ the abelian group~$\Integer/n$ has the projective resolution
\[
  \underbrace{
  \dotsb
  \to
  0
  \to
  0
  \to
  \Integer
  \xlongto{n}
  \Integer
  }_{\Pcc}
  \xlongto{p_0}
  \Integer/n
  \to
  0 \,.
\]
The resulting chain complex
\[
  \Hom_\Integer(\Pcc, \Integer/m)
  =
  \bigl(
    \Hom_\Integer(\Integer,\Integer/m)
    \xlongto{n}
    \Hom_\Integer(\Integer,\Integer/m)
    \to
    0
    \to
    0
    \to
    \dotsb
  \bigr)
\]
is isomorphic to the chain complex
\[
  \Integer/m
  \xlongto{n}
  \Integer/m
  \to
  0
  \to
  0
  \to
  \dotsb
\]
By taking the zeroeth cohomology of this cochain complex we find that
\begin{align*}
  ( \Right^0 \Hom_\Integer(-,\Integer/m) )(\Integer/n)
  &\cong
  \{
  x \in \Integer/m
  \suchthat
  nx = 0  
  \}
  \\
  &=
  \text{\dash{$n$}{torsion} of~$\Integer/m$}
  \\
  &=
  \text{\dash{$\gcd(n,m)$}{torsion} of~$\Integer/m$}
  \cong
  \begin{cases}
    \Integer/\gcd(n,m)  & \text{if~$m \neq 0$}  \,, \\
    0                   & \text{if~$m = 0$}     \,.
  \end{cases}
\end{align*}
By taking the first cohomology we find that
\[
  ( \Right^1 \Hom_\Integer(-,\Integer/m) )(\Integer/n)
  \cong
  n \cdot \Integer/m
  =
  \gcd(n,m) \cdot \Integer/m
  \cong
  \Integer/\lcm(n,m) \,.
\]
We also find that
\[
  ( \Right^i \Hom_\Integer(-,\Integer/m) )(\Integer/n)
  = 0
\]
for every~$i \geq 2$.





\subsection{}

For~$n = 0$ the abelian group~$\Integer/n = \Integer$ has the projective resolution
\[
  \underbrace{
  \dotsb
  \to
  0
  \to
  0
  \to
  \Integer
  }_{\Pcc}
  \xlongto{\id}
  \Integer
  \to
  0 \,.
\]
The resulting chain complex
\[
  \Pcc \tensor_\Integer (\Integer/m)
  =
  \bigl(
  \dotsb
  \to
  0
  \to
  0
  \to
  \Integer \tensor_\Integer (\Integer/m)
  \bigr)
\]
is isomophic to the chain complex
\[
  \dotsb
  \to
  0
  \to
  0
  \to
  \Integer/m \,.
\]
By taking the homology of this chain complex we find that
\[
  ( \Left^i (- \tensor_\Integer (\Integer/m)) )(\Integer)
  \cong
  \begin{cases}
    \Integer/m  & \text{for~$i = 0$}  \,, \\
    0           & \text{for~$i > 0$}  \,.
  \end{cases}
\]
(One can also proceed as above, using that~$\Left^0 (- \tensor_\Integer (\Integer/m)) \cong (- \tensor_\Integer (\Integer/m))$ and that~$(\Left^i (- \tensor_\Integer (\Integer/m)))(\Integer) = 0$ for~$i > 0$ because~$\Integer$ is free abelian and hence projective in~$\Ab$.

For~$n \neq 0$ the abelian group~$\Integer/n$ has the projective resolution
\[
  \underbrace{
  \dotsb
  \to
  0
  \to
  0
  \to
  \Integer
  \xlongto{n}
  \Integer
  }_{\Pcc}
  \xlongto{p_0}
  \Integer/n
  \to
  0 \,.
\]
The resulting chain complex
\[
  \Pcc \tensor_\Integer (\Integer/m)
  =
  \bigl(
    \dotsb
    \to
    0
    \to
    0
    \to
    \Integer \tensor_\Integer (\Integer/m)
    \xlongto{n}
    \Integer \tensor_\Integer (\Integer/m)
  \bigr)
\]
is isomorphic to the chain complex
\[
  \dotsb
  \to
  0
  \to
  0
  \to
  \Integer/m
  \xlongto{n}
  \Integer/m \,.
\]
We find as before that
\[
  ( \Left^0 (- \tensor_\Integer (\Integer/m)) )(\Integer/n)
  \cong
  n \cdot \Integer/m
  =
  \gcd(n,m) \cdot \Integer/m
  \cong
  \Integer/\lcm(n,m)
\]
and
\begin{align*}
  ( \Left^1 (- \tensor_\Integer (\Integer/m)) )(\Integer/n)
  &\cong
  \{
  x \in \Integer/m
  \suchthat
  nx = 0  
  \}
  \\
  &=
  \text{\dash{$n$}{torsion} of~$\Integer/m$}
  \\
  &\cong
  \begin{cases}
    \Integer/\gcd(n,m)  & \text{if~$m \neq 0$}  \,, \\
    0                   & \text{if~$m = 0$}     \,,
  \end{cases}
\end{align*}
as well as
\[
  ( \Right^i \Hom_\Integer(-,\Integer/m) )(\Integer/n)
  = 0
\]
for every~$i \geq 2$.




