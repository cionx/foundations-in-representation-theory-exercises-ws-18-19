\section{}


We observe that~$A \cong \Integer[\Integer/n]$ is a free~\dash{$\Integer$}{module} with basis~$1, t, \dotsc, t^{n-1}$.
We will denote the powers of~$t$ in~$A$ as both~$t^i$ with~$i \in \Natural$ and~$t^i$ with~$i \in \Integer/n$, where~$t^{[i]} \defined t^i$ for every~$[i] \in \Integer/n$.
For such a linear combination~$\sum_{i=0}^{n-1} a_i t^i$ with~$a_i \in \Integer$ we also set~$a_{[i]} \defined a_i$ for every~$[i] \in \Integer/n$, so that
\[
  \sum_{i=0}^{n-1} a_i t^i
  =
  \sum_{i \in \Integer/n} a_i t^i \,.
\]





\subsection{}

We have
\[
  q (1-t)
  =
  (1 + t + \dotsb + t^{n-1}) (1-t)
  =
  1 - t^n
  =
  0
\]
which shows that
\[
  \dotsb
  \to
  A
  \xlongto{q}
  A
  \xlongto{1-t}
  A
  \xlongto{q}
  A
  \xlongto{1-t}
  A
\]
is a complex.
The~{\module{$A$}}~$A$ is free and hence projective.





\subsection{}

We have for every~$x \in A$ with~$x = \sum_{i=0}^{n-1} a_i t^i$ that
\begin{align*}
  qx
  =
  \sum_{i \in \Integer/n} t^i
  \sum_{j \in \Integer/n} a_j t^j
  &=
  \sum_{i,j \in \Integer/n} a_j t^{i+j}
  =
  \sum_{j \in \Integer/n} a_j \sum_{i \in \Integer/n} t^{i+j}
  \\
  &=
  \sum_{ j \in \Integer/n } a_j
  \sum_{ k \in \Integer/n } t^k
  =
  \left( \sum_{ j \in \Integer/n } a_j \right)
  q \,.
\end{align*}
It follows that
\[
  qx = 0
  \iff
  \sum_{i=0}^{n-1} a_i = 0 \,.
\]
This means that for~$qx = 0$ we may write the tupel~$(a_0, \dotsc, a_{n-1}) \in \Integer^n$ as a linear combination of~$e_0 - e_1, \dotsc, e_{n-2} - e_{n-1}$, say
\[
  (a_0, \dotsc, a_{n-1})
  =
  b_0 (e_0 - e_1) + \dotsb + b_{n-2} (e_{n-2} - e_{n-1}) \,.
\]
It follows that
\[
  x
  =
  \sum_{i=0}^{n-1} a_i t^i
  =
  \sum_{j=0}^{n-2} b_j (t^j - t^{j+1})
  =
  \sum_{j=0}^{n-2} b_j t^j (1-t)  \,.
\]
This shows the exactness at~$P_{2n}$ for~$n > 0$.

We also have that
\begin{align*}
  (1-t)x
  &=
  (1-t) \sum_{i \in \Integer/n} a_i t^i
  =
  \sum_{i \in \Integer/n} a_i t^i - \sum_{i \in \Integer/n} a_i t^{i+1}
  \\
  &=
  \sum_{i \in \Integer/n} a_i t^i - \sum_{i \in \Integer/n} a_{i-1} t^i
  =
  \sum_{i \in \Integer/n} (a_i - a_{i-1}) t^i
\end{align*}
and therefore that
\[
  (1-t)x = 0
  \iff
  a_0 = \dotsb = a_{n-1} \,.
\]
This shows that for~$(1-t)x = 0$ there exists some~$a \in \Integer$ with~$x = a q$, which gives the exactness at~$P_{2n+1}$ for~$n \geq 0$.





\subsection{}

The map~$p_0$ is surjective because~$p_0(n) = n$ for every~$n \in \Integer$.
It is also~\dash{$A$}{linear} because it is additive (i.e.~\dash{$\Integer$}{linear}) with
\[
  p_0(tx)
  =
  p_0(x)
  =
  t p_0(x)
\]
for every~$x \in A$.

We have for~$x = \sum_{i \in \Integer/n} a_i t^i \in A$ that~$x \in \ker(p_0)$ if and only if~$\sum_{i \in \Integer/n} a_i = 0$.
We have seen above that this is equivalent to~$qx = 0$.
The exactness at~$P_0$ does therefore follow from the exactness of~$A \xto{1-t} A \xto{q} A$.





\subsection{}

We use the constructed projective resolution
\[
  \underbrace{
  \dotsb
  \to
  A
  \xlongto{q}
  A
  \xlongto{1-t}
  A
  \xlongto{q}
  A
  \xlongto{1-t}
  A
  }_{\Pcc}
  \xlongto{p_0}
  \Integer
  \to
  0 \,.
\]
By applying the functor~$- \otimes_A \Integer$ we get the chain complex
\[
  \Pcc \otimes_A \Integer
  =
  \bigl(
  \dotsb
  \to
  A \otimes_A \Integer
  \xlongto{q}
  A \otimes_A \Integer
  \xlongto{1-t}  
  A \otimes_A \Integer
  \xlongto{q}
  A \otimes_A \Integer
  \xlongto{1-t}
  A \otimes_A \Integer
  \bigr)  \,,
\]
which is isomorphic to the chain complex
\[
  \dotsb
  \to
  \Integer
  \xlongto{n}
  \Integer
  \xlongto{0}
  \Integer
  \xlongto{n}
  \Integer
  \xlongto{0}
  \Integer \,.
\]
By taking the homology of this chain complex we find that
\[
  (\Left^i (- \otimes_A \Integer))(\Integer)
  \cong
  \begin{cases}
    \Integer    & \text{for~$i = 0$}  \,, \\
    \Integer/n  & \text{for~$i \geq 1$ odd} \,, \\
    0           & \text{for~$i \geq 2$ even}  \,.
  \end{cases}
\]
