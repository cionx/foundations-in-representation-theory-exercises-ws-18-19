\section{}





\subsection{}

We can use the (very explicit) standard projective resolution (from Theorem~6.11).





\subsection{}

The short exact sequence
\[
  0
  \to
  P(2)^{\oplus 2}
  \to
  P(1) \oplus P(2)^{\oplus 2}
  \to
  M
  \to
  0
\]
results in a long exact sequence
\[
  \begin{tikzcd}[column sep = 0.9em, cramped]
      0
      \arrow{r}
    & \End(M)
      \arrow{r}
      \arrow[d, phantom, ""{coordinate, name=Z}]
    & \Hom(P(1) \oplus P(2)^{\oplus 2}, M)
      \arrow{r}
    & \Hom(P(2)^{\oplus 2}, M)
      \arrow[ dll,
              rounded corners,
              to path={ -- ([xshift=2ex]\tikztostart.east)
                        |- (Z) \tikztonodes
                        -| ([xshift=-2ex]\tikztotarget.west)
                        -- (\tikztotarget)}
            ]
    \\
      {}
    & (\Right^1 \Hom(-,M))(M)
      \arrow{r}
    & (\Right^1 \Hom(-,M))(P(1) \oplus P(2)^{\oplus 2})
      \arrow{r}
    & \cdots
  \end{tikzcd}
\]
The representation~$P(1) \oplus P(2)^{\oplus 2}$ is again projective, whence
\[
  (\Right^1 \Hom(-,M))(P(1) \oplus P(2)^{\oplus 2})
  =
  0 \,.
\]
We therefore have the exact sequence
\[
  \begin{tikzcd}[column sep = 1em, row sep = small]
      0
      \arrow{r}
    & \End(M)
      \arrow{r}
      \arrow[d, phantom, ""{coordinate, name=Z1}]
    & \Hom(P(1) \oplus P(2)^{\oplus 2}, M)
      \arrow{r}
    & \Hom(P(2)^{\oplus 2}, M)
      \arrow[ dll,
              rounded corners,
              to path={ -- ([xshift=2ex]\tikztostart.east)
                        |- (Z1) \tikztonodes
                        -| ([xshift=-2ex]\tikztotarget.west)
                        -- (\tikztotarget)}
            ]
    \\
      {}
    & (\Right^1 \Hom(-,M))(M)
      \arrow{r}
    & 0
    & {}
  \end{tikzcd}
\]
It follows that
\begin{align*}
  {}&
  \dim {(\Right^1 \Hom(-,M))(M)}
  \\
  ={}&
  \dim \Hom(P(2)^{\oplus 2}, M)
  -
  \dim \Hom(P(1) \oplus P(2)^{\oplus 2}, M)
  +
  \dim
  \End(M)
  \\
  ={}&
  2 \dim M_2
  -
  (\dim M_1 + 2 \dim M_2)
  +
  \dim \End(M)
  \\
  ={}&
  \dim \End(M) - 1 \,,
\end{align*}
where we use that
\[
  \Hom(N_1 \oplus N_2, M)
  \cong
  \Hom(N_1, M) \oplus \Hom(N_2, M)
\]
for any two representations~$N_1$ and~$N_2$, and
\[
  \Hom(P(i), M)
  \cong
  M_i
\]
for every~$i \in Q_0$.



