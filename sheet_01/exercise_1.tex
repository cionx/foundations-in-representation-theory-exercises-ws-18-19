\section{}

We convince ourselves that~$A^\times$ is a group with respect to the multiplication of~$A$. 

\begin{itemize}
  \item
    If~$a_1, a_2 \in A^\times$ are units in~$A$ then there exist~$b_1, b_2 \in A$ with~$a_i b_i = 1 = b_i a_i$ for~$i = 1,, 2$.
    It then follows that
    \[
        (a_1 a_2) (b_2 b_1)
      = a_1 a_2 b_2 b_1
      = a_1 \cdot 1 \cdot b_1
      = a_1 b_1
      = 1
    \]
    and similarly~$(b_2 b_1) (a_1 a_2) = 1$.
    This shows that~$a_1, a_2$ is again a unit, which shows that~$A^\times$ is closed under the multiplication of~$A$.
  \item
    It follows from the equation~$1 \cdot 1 = 1$ that~$1 \in A$ is a unit with inverse~$1^{-1}$, and that~$1$ is a neutral element for~$A^\times$.
  \item
    If~$a \in A^\times$ is a unit with~$ab = 1 = ba$ for some~$b \in A$ then~$b$ is again a unit and an inverse for~$a$ in~$A^\times$.
\end{itemize}

If~$f \colon A \to B$ is a {\kalg} homomorphism then it holds for~$a \in A^\times$ that
\[
    f(a) f(a^{-1})
  = f(a a^{-1})
  = f(1)
  = 1
\]
and similarly~$f(a^{-1}) f(a) = 1$, which shows that~$f(a)$ is a unit in~$B$.
This shows that~$f$ restricts to a map~$f \colon A^\times \to B^\times$, which is then still multiplicative and therefore a group homomorphism.





\subsection{}

It follows from~$G$ being a {\kbasis} of~$k[G]$ that there exists a unique~{\klin}~extension~$f \colon k[G] \to A$ with~$f(g) = \psi(g)$ of~$\psi$.
We onl need to show that~$f$ is already a homomorphism of~{\kalg}:
The map~$f$ is multiplicative on the~{\kbasis}~$G$ of~$k[G]$ and therefore multiplicativeon the whole of~$k[G]$.
It also holds that~$f(1) = \psi(1) = 1$.





\subsection{}

We denote the basis element of~$k[\Integer^n]$ associated to the group element~$(a_1, \dotsc, a_n) \in \Integer^n$ by~$t_1^{a_1} \dotsm t_n^{a_n}$.
The group algebra~$k[\Integer^n]$ has then a basis~$t_1^{a_1} \dotsm t_n^{a_n}$ with~$a_1, \dotsc, a_n \in \Integer$ on which the~{\kbil} multiplication of~$k[\Integer^n]$ is given by
\[
        t_1^{a_1} \dotsm t_n^{a_n}
  \cdot t_1^{b_1} \dotsm t_n^{b_n}
  =     t_1^{a_1 + b_1} \dotsm t_n^{a_n + b_n} \,.
\]
This shows that~$k[\Integer^n] \cong k[t_1, t_1^{-1}, \dotsc, t_n, t_n^{-1}]$ is the~{\kalg} of Lauraunt polynomials in~$n$ variables~$t_1, \dotsc, t_n$ over~$k$.

The group algebra~$k[\Integer/2]$ has a~{\kbasis} consisting of two elements~$1, s$, with~$1$ being the unit of~$A$ and~$s$ satisfying the equation~$s^2 = 1$.
This shows that~$k[\Integer/2] \cong k[x]/(x^2 - 1)$.
If~$\ringchar(k) = 2$ then it follows that
\[
        k[\Integer/2]
  \cong k[x]/(x^2 - 1)
  =     k[x]/((x-1)^2)
  \cong k[x]/(x^2)
\]
is the algebra of dual numbers.
If~$\ringchar(k) = 2$ then it follows from the chinese remainder theorem that
\begin{align*}
          k[\Integer/2]
   \cong  k[x]/(x^2 - 1)
  &=      k[x]/((x+1)(x-1)) \\
  &\cong  k[x]/(x+1) \times k[x]/(x-1)
   \cong  k \times k \,.
\end{align*}


