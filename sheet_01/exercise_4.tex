\section{}





\subsection{}
\label{ideals in matrix rings}


We give two solutions.



\subsubsection{First Solution}


\begin{lemma}
  Let~$R$ be a ring and let~$n \geq 1$.
  Then the map
  \[
            \{ \text{{\twosided} ideals~$I \subseteq I$} \}
    \longto \{ \text{{\twosided} ideals~$J \subseteq \mat{n}{R}$} \} \,,
    \quad   I
    \mapsto \mat{n}{I}
  \]
  is a {\welldef} bijection.
\end{lemma}


\begin{proof}
  If~$I \subseteq R$ is a {\twosided} ideal then the canonical projection~$\pi \colon R \to R/I$ is a ring homomorphism with~$\ker(\pi) = I$.
  The kernel of the induced ring homomorphism~$\pi_* \colon \mat{n}{R} \to \mat{n}{R/I}$ is given by~$\ker(\pi_*) = \mat{n}{I}$, which shows that~$\mat{n}{I}$ is a {\twosided} ideal in~$\mat{n}{R}$.
  This shows that the proposed map is {\welldef}.
  The map is injective because one can retrieve~$I$ from~$\mat{n}{I}$ as the set of all possible entries of matrices in~$\mat{n}{I}$.
  
  To show the surjectivity let~$J \subseteq \mat{n}{R}$ be a {\twosided} ideal.
  For all~$i,j = 1, \dotsc, n$ let
  \[
      I_{ij}
    = \{
        a \in R
      \suchthat
        \text{there exists a matrix~$A \in J$ whose~$(i,j)$-th entry is~$a$}
      \} \,.
  \]
  If~$\pi_{ij} \colon \mat{n}{R} \to R$ denotes the projection onto the~$(i,j)$-th component then~$\pi_{ij}$ is a homomorphism of both left~$R${\dashmods} and of right~$R${\dashmods}.
  It follows that~$I_{ij} = \pi_{ij}(I)$ is both a left~$R${\dashsmod} of~$R$ and a right~$R${\dashsmod} of~$R$, and therefore a {\twosided} ideal in~$R$.
  
  For~$A \in J$ and~$i, j = 1, \dotsc, n$ we can multiply~$A$ from the left and from the right with permuation matrices to move the~$(i,j)$-th entry of~$A$ to any other position.
  We don’t leave~$J$ when doing so because it is a {\twosided} ideal.
  This shows that the {\twosided} does not depend on the choice of position~$(i,j)$, i.e.\ that there exists a unique {\twosided} ideal~$I \subseteq R$ with~$\pi_{ij}(J) = I$ for all~$i,j$.
  It follows from the construction of~$I$ that~$J = \mat{n}{I}$.
\end{proof}


We now get the statement from the exercise for free:
If~$n = 0$ then~$\mat{n}{k} = 0$ and there exist no nonzero {\twosided} ideals.
If~$n \geq 1$ then the~{\twosided} ideals in~$\mat{n}{k}$ correspond bijectively to the {\twosided} ideals in~$k$, of which there are only two.
It then follows from~$\mat{n}{k} \neq 0$ that~$\mat{n}{k}$ contains no proper nonzero ideal.



\subsubsection{Second Solution}

Let~$I \subseteq \mat{n}{k}$ be a nonzero ideal.
There then exists some matrix~$A \in I$ whose~$(i,j)$-th entry is nonzero for suitable~$i,j$.
It follows with~$c \defined A_{ij} \neq 0$ that
\[
      E_{ij}
  =   \frac{1}{c} E_{ii} A E_{jj}
  \in I \,.
\]
By multiplying~$E_{ij}$ from both sides with permutation matrices it then follows that~$I$ contains the standard basis of~$\mat{n}{k}$.
This shows that~$I = \mat{n}{k}$ because~$I$ is a~$k${\dashsmod} of~$\mat{n}{k}$.





\subsection{}

We give three solutions.



\subsubsection{First Solution}

Let~$M$ be a nonzero~{\Amod} and let~$\varphi \colon A \to  \End_k(M)$ be the associated {\kalg} homomorphism.
The homomorphism~$\varphi$ is nonzero since its image contains the identity~$\id_M$.
This shows that the kernel~$\ker(\varphi)$, which is a {\twosided} ideal in~$A$, is a proper ideal.
It follows from part~\ref{ideals in matrix rings} of the exercise that~$\ker(\varphi) = 0$ and therefore that~$\varphi$ is injective.
It follows that
\[
        \dim_k  \End_k(V)
  \geq  \dim_k  A
  =             n^2
\]
and therefore that~$\dim_k M \geq n$.



\subsubsection{Second Solutions}

Recall from algebra the~{\kalg}~$A$ is semisimple, and that~$k^n$ is up to isomorpism the only simple~{\Amod}.
It follows that every~{\Amod}~$M$ is of the form~$M = (k^n)^{\oplus I}$ for some index set~$I$, and thus in particular that~$\dim_k M \geq \dim_k k^n = n$ if~$M$ is nonzero.



\subsubsection{Third Solution}

We can also give a more ad hoc version of the above argument:
For every~$i = 1, \dotsc, n$ let~$C_i = A E_{ii}$, which is the left ideal of~$A$ which consist of all matrices for which all columns except for the~$i$-th one vanish.
It then holds that~$A = C_1 \oplus \dotsb \oplus C_n$.
It holds for every~$i$ that~$C_i \cong k^n$ as~{\Amod}, and~$k^n$ is simple as an~{\Amod} (because every nonzero element~$x \in k^n$ generates~$k^n$ as an~{\Amod}).

Let~$M$ be a nonzero~{\Amod} and let~$m \in M$ be a nonzero element.
The map
\[
          \varphi
  \colon  A
  \to     M \,,
  \quad   a
  \mapsto am
\]
is then a nonzero homomorphism of~{\Amod}
It follows from~$A = C_1 + \dotsb + C_n$ that the kernel~$\ker(\varphi)$ does not contain on of the~$C_i$.
It then follows that~$C_i \cap \ker(\varphi)$ is a proper {\Asmod} of~$C_i$, and therefore that~$C_i \cap \ker(\varphi) = 0$.
This shows that the restriction~$\restrict{\varphi}{C_i} \colon C_i \to M$ is injective, from which it follows that
\[
        \dim_k M
  \geq  \dim_k C_i
  =     \dim_k k^n
  =     n \,.
\]




