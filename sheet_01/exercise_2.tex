\section{}

In the given quiver~$Q$ there exists for every pair~$(i,j)$ of vertices~$i,j \in \{1, \dotsc, n\} = Q_0$ a unique path from~$i$ to~$j$ in~$Q$ if~$i \leq j$, and no path otherwise.
It follows that the path algebra~$kQ$ has a basis~$e_{ij}$ with~$1 \leq j \leq i \leq n$, where~$e_{ij}$ is the unique path from~$j$ to~$i$ in~$Q$.
It holds that
\[
    e_{ij} e_{kl}
  = \delta_{jk} e_{il}
\]
for all~$i,j,k,l$ by definition of the multiplication of~$kQ$.

If~$E_{ij}$ with~$1 \leq j \leq i \leq n$ denotes the standard basis of~$L_n$ then it holds that
\[
    E_{ij} E_{kl}
  = \delta_{jk} E_{il}
\]
for all~$i,j,k,l$.
The unique~{\klin} map~$\varphi \colon kQ \to L_n$ with~
\[
    \varphi(e_{ij})
  = E_{ij}
\]
for all~$1 \leq j \leq i \leq n$ is an isomorphism because it maps the above basis of~$kQ$ bijectively onto the standard basis of~$L_n$.
The isomorphism~$\varphi$ is multiplicative on a~\dash{$k$}{generating} set of~$kQ$, and therefore multiplicative on all of~$Q$.

This shows that~$\varphi$ is multiplicative and~{\klin}, which shows that it is an isomorphism of~{\kalgs}.
(That~$\varphi(1) = 1$ follows from the multiplicativity and bijectivity.)




