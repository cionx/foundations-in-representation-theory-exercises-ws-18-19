\section{}

In the given quiver~$Q$ there exists for every pair~$(i,j)$ of vertices~$i,j \in \{1, \dotsc, n\} = Q_0$ a unique path~$e_{ij}$ from~$i$ to~$j$ in~$Q$ if~$i \leq j$, and no path otherwise.
The paths~$e_{ij}$ with~$1 \leq i \leq j \leq n$ are by definition a~{\kbasis} of~$kQ$.

The quiver~$Q$ has a representation over~$k$ with vertices~$k$ and edges~$k \xto{1} k$:
\[
  \begin{tikzcd}
      k
      \arrow{r}[above]{1}
    & k
      \arrow{r}[above]{1}
    & k
      \arrow{r}[above]{1}
    & \cdots
      \arrow{r}[above]{1}
    & k
      \arrow{r}[above]{1}
    & k
  \end{tikzcd}
\]
This representation of~$Q$ over~$k$ corresponds to a~{\module{$kQ$}} structure on~$k^{\oplus n} = k^n$ as discussed in the lecture (i.e.\ as in the standard way).
The action of~$e_{ij}$ on~$k^n$ is with respect to the standard basis given by the lower triangular matrix~$E_{ji}$.
The~{\kalg} homomorphism~$\varphi \colon kQ \to \End_k(k^n) = \mat{n}{k}$ which corresponds to this~{\module{$kQ$}} structure therefore maps~$e_{ij}$ onto~$E_{ji}$.
The homomorphisms~$\varphi$ therefore maps the path~{\kbasis} of~$kQ$ bijectively onto the standard~{\kbasis} of~$L_n$, and it is hence an isomorphism of~{\kalgs}.

% 
% 
% It follows that the path algebra~$kQ$ has a~{\kbasis}~$e_{ij}$ with~$1 \leq j \leq i \leq n$, where~$e_{ij}$ is the unique path from~$j$ to~$i$ in~$Q$.
% It holds that
% \[
%     e_{ij} e_{kl}
%   = \delta_{jk} e_{il}
% \]
% for all~$i,j,k,l$ by definition of the multiplication of~$kQ$.
% 
% If~$E_{ij}$ with~$1 \leq j \leq i \leq n$ denotes the standard~{\kbasis} of~$L_n$ then it holds that
% \[
%     E_{ij} E_{kl}
%   = \delta_{jk} E_{il}
% \]
% for all~$i,j,k,l$.
% The unique~{\klin} map~$\varphi \colon kQ \to L_n$ with~
% \[
%     \varphi(e_{ij})
%   = E_{ij}
% \]
% for all~$1 \leq j \leq i \leq n$ is an isomorphism because it maps the above~{\kbasis} of~$kQ$ bijectively onto the standard~{\kbasis} of~$L_n$.
% The isomorphism~$\varphi$ is multiplicative on a~\dash{$k$}{generating} set of~$kQ$, and therefore multiplicative on all of~$Q$.
% 
% This shows that~$\varphi$ is multiplicative and~{\klin}, which shows that it is an isomorphism of~{\kalgs}.
% (That~$\varphi(1) = 1$ follows from the multiplicativity and bijectivity.)
% 



