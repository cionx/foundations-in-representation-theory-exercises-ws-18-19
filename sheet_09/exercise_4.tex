\section{}





\subsection{}

The components of the morphism~$fh \colon \cone(f)[1] \to \Dcc$ are given by
\[
    (fh)_n
  = f_n h_n
  = f_n
    \begin{bmatrix}
      -\id  & 0
    \end{bmatrix}
  = \begin{bmatrix}
      -f_n  & 0
    \end{bmatrix}
\]
for every~$n \in \Integer$.
The collection~$s = (s_n)_{n \in \Integer}$ of morphisms
\[
            s_n
  \defined  \begin{bmatrix}
              0 & -\id
            \end{bmatrix}
  \colon    C_n \oplus D_{n+1}
  \to       D_{n+1}
\]
are a null homotopy for~$fh$ because
\begin{align*}
      d_{n+1} s_n + s_{n-1} d_n
  &=  d_{n+1}
      \begin{bmatrix}
        0 & -\id
      \end{bmatrix}
      +
      \begin{bmatrix}
        0 & -\id
      \end{bmatrix}
      \begin{bmatrix}
        d_n &           \\
        f_n & -d_{n+1}
      \end{bmatrix}
\\
  &=  \begin{bmatrix}
        0 & -d_{n+1}
      \end{bmatrix}
      \begin{bmatrix}
        -f_n  & d_{n+1}
      \end{bmatrix}
   =  \begin{bmatrix}
        -f_n  & 0
      \end{bmatrix}
   =  (fh)_n
\end{align*}
for every~$n \in \Integer$.





\subsection{}

We know from the previous part of the exercise that~$hf$ is null homotpic in~$\Ch(\Acat)$, hence that~$hf = 0$ in~$\Homotopy(\Acat)$.
It follows that~$h = 0$ in~$\Homotopy(\Acat)$ because~$f$ is a monomorphism in~$\Acat$, hence that~$h$ in null homotopic in~$\Ch(\Acat)$.





\subsection{}

Let~$s$ be a null homotopy for the morphism~$h$, i.e.\ let~$s = (s_n)_{n \in \Integer}$  be a family of morphisms~$s_n \colon C_n \oplus D_{n+1} \to C_{n+1}$ with
\[
    h_n
  = d_{n+1} s_n + s_{n-1} d_n
\]
for every~$n \in \Integer$.
We may write every~$s_n$ as
\[
    s_n
  = \begin{bmatrix}
      t_n & -u_{n+1}
    \end{bmatrix}
\]
for some morphisms~$t_n \colon C_n \to C_{n+1}$ and~$u_n \colon D_{n+1} \to C_{n+1}$.
Then
\begin{align*}
      \begin{bmatrix}
        -\id  & 0
      \end{bmatrix}
  &=  h_n \\
  &=  d_{n+1} s_n + s_{n-1} d_n \\
  &=  d_{n+1}
      \begin{bmatrix}
        t_n & -u_{n+1}
      \end{bmatrix}
      +
      \begin{bmatrix}
        t_{n-1} & u_n
      \end{bmatrix}
      \begin{bmatrix}
        d_n &           \\
        f_n & -d_{n+1}
      \end{bmatrix} \\
  &=  \begin{bmatrix}
        d_{n+1} t_n & -d_{n+1} u_{n+1}
      \end{bmatrix}
      \begin{bmatrix}
        t_{n-1} d_n - u_n f_n & u_n d_{n+1}
      \end{bmatrix} \\
  &=  \begin{bmatrix}
        d_{n+1} t_n + t_{n-1} d_n - u_n f_n & u_n d_{n+1} - d_{n+1} u_{n+1}
      \end{bmatrix} \,.
\end{align*}
We hence have for every~$n \in \Integer$ that
\[
  \left\{
    \begin{array}{rcl}
        u_n f_n - \id                 &=& d_{n+1} t_n + t_{n-1} d_n \,, \\
        u_n d_{n+1} - d_{n+1} u_{n+1} &=& 0 \,.
    \end{array}
  \right.
\]
The second conditions tells us that~$u \defined (u_n)_{n \in \Integer}$ is a morphism of chain complexes~$u \colon \Dcc \to \Ccc$, and the first condition states that~$uf - \id$ is null homotopic in~$\Ch(\Acat)$, i.e.\ that~$uf = \id$ in the category~$\Homotopy(\Acat)$.





\subsection{}

Parts~(ii) and~(iii) show that every monomorphism in~$\Homotopy(\Acat)$ splits, hence that every kernel in~$\Homotopy(\Acat)$ splits.
Suppose that a kernel~$k \colon K \to \Integer/4$ of~$g$ in~$\Homotopy(\Acat)$ exists.

The inclusion~$i \colon \Integer/2 \to \Integer/4$ satisfies~$gi = 0$, hence factors through~$k$ in~$\Homotopy(\Acat)$:
\[
  \begin{tikzcd}
      K
      \arrow{r}[above]{k}
    & \Integer/4
      \arrow{r}[above]{g}
    & \Integer/2
    \\
      {}
    & \Integer/2
      \arrow[dashed]{ul}
      \arrow{u}[right]{i}
    & {}
  \end{tikzcd}
\]
We get in the zeroeth homology the following induced commutative diagram:
\[
  \begin{tikzcd}[column sep = large]
      \Hl_0(K)
      \arrow{r}[above]{\Hl_0(k)}
    & \Integer/4
      \arrow{r}[above]{g}
    & \Integer/2
    \\
      {}
    & \Integer/2
      \arrow[dashed]{ul}
      \arrow{u}[right]{i}
    & {}
  \end{tikzcd}
\]
It follows from the commutativity of the triangle that~$\Hl_0(K) \neq 0$.

We also have that~$\Hl_0(k) \colon \Hl_0(K) \to \Integer/4$ is a section because~$k$ is a section.
It follows from~$\Integer/4$ being indecomposable that either~$\Hl_0(k) = 0$ or~$\Hl_0(k)$ is an isomorphism.
We have just seen that~$\Hl_0(k) \neq 0$, so~$\Hl_0(k)$ is an isomorphism.

But we get by taking the zeroeth homology of the commutative diagram
\[
  \begin{tikzcd}
      K
      \arrow{r}[above]{k}
      \arrow[bend right]{rr}[below]{0}
    & \Integer/4
      \arrow{r}[above]{g}
    & \Integer/2
  \end{tikzcd}
\]
the following commutative diagram:
\[
  \begin{tikzcd}[column sep = large]
      \Hl_0(K)
      \arrow{r}[above]{\Hl_0(k)}
      \arrow[bend right]{rr}[below]{0}
    & \Integer/4
      \arrow{r}[above]{g}
    & \Integer/2
  \end{tikzcd}
\]
This contradicts~$\Hl_0(k)$ being an isomorphism because~$g \neq 0$.










