\section{}

We denote the given category by~$\Gcat$.





\subsection{}

A~\dash{$G$}{set}~$X$ consists of a set~$X$ and for every~$g \in G$ a map~$X_g \colon X \to X$, in such a way that~$X_e = \id_X$ and~$X_{g_1 g_2} = X_{g_1} \circ X_{g_2}$ for all~$g_1, g_2 \in G$.
A functor~$F \colon \Gcat$ consists of a single set~$F(\ast)$ and for every~$g \in G = \Gcat(\ast,\ast)$ a map~$F(g) \colon F(\ast) \to F(\ast)$, in such a way that~$F(e) = \id_X$ and~$F(g_1 g_2) = F(g_1) F(g_2)$ for all~$g_1, g_2 \in G$.
We see by setting~$X = F(\ast)$ and~$X_g = F(gr$ that both concept consist of precisely the same data.
This shows that a~\dash{$G$}{set}~$X$ is the same as a functor~$F \colon \Gcat \to \Set$.

Let~$X$ and~$X'$ be two~\dash{$G$}{sets} with corresponding functors~$F, F' \colon \Gcat \to \Set$, and let~$f \colon X \to X'$ be a map.
Then
\begin{align*}
      {}& \text{$f$ is a homomorphism of~\dash{$G$}{sets}}  \\
  \iff{}& \text{$f(gx) = g(f(x))$ for all~$g \in G$,~$x \in X$} \\
  \iff{}& \text{$f \circ F(g) = F'(g) \circ f$ for every~$g \in G$} \\
  \iff{}& \text{$f \colon F(\ast) \to F'(\ast)$ defines a natural transformation~$F \to F'$}
\end{align*}

This altogether shows that the category~$\Gset{G}$ is isomorphic to the functor category~$\Fun(\Gcat, \Set)$, and therefore in particular equivalent to it.





\subsection{}

The~\dash{$G$}{set}~$X$ which corresponds to the covariant functor~$h^\ast \colon \Gcat \to \Set$ is given by the set~$X = h^\ast(\ast) = \Gcat(\ast,\ast) = G$, and the action of~$g \in G$ on~$x \in G$ is given by
\[
    gx
  = h^\ast(g)(x)
  = g_*(x)
  = g \circ x
  = gx \,.
\]
The~\dash{$G$}{set}~$X$ is therefore just the regular left~\dash{$G$}{set}, i.e.\ the group~$G$ acting on itself by left multiplication.





\subsection{}

Let~$X$ be the left regular~\dash{$G$}{set}, as before.
We find that
\[
        \End_{\Gset{G}}(X)
  =     \Gset{G}(X,X)
  \cong \Fun(\Gset{G}, \Set)(h^\ast, h^\ast)
  \cong h^\ast(\ast)
  =     \Gcat(\ast,\ast)
  =     G \,.
\]
Under this bijection, the group element~$g \in G$ corresponds the the endomorphism of~\dash{$G$}{sets}~$X \to X$ which is given by right multiplication of~$g$.
Thus the Yoneda lemma tells us that every endomorphism of~$X$ is given by right multiplication with a unique element~$g \in G$.
Moreover, the explicit description of the bijection
\[
          \Fun(\Gset{G}, \Set)(h^\ast, h^\ast)
  \to     h^\ast(\ast) \,,
  \quad   \eta
  \mapsto \eta_\ast(\ast)
\]
tells us that for every endomorphism of~\dash{$G$}{sets}~$\varphi \colon X \to X$, the corresponding group element~$g \in G$ is given by~$g = \varphi(e)$.




