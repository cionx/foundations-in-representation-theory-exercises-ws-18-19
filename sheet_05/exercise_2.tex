\section{}




\subsection{}

\begin{lemma}
  \label{adjoint mono epi}
  Let~$(F, G, \varphi)$ be an adjunction between categories~$\Ccat$ to~$\Dcat$.
  Then~$F$ preserves epimorphisms and~$G$ preserves monomorphisms, i.e.\ if~$f$ is an epimorphism in~$\Ccat$ then~$F(f)$ is an epimorphism in~$\Dcat$, and if~$g$ is a monomorphism in~$\Dcat$ then~$G(g)$ is a monomorphism in~$\Ccat$.
\end{lemma}

\begin{proof}
      For every morphism~$f \colon X \to X'$ in~$\Ccat$ and every object~$Y \in \Ob(\Dcat)$ the square
      \begin{equation}
        \label{first diagram}
        \begin{tikzcd}[sep = large]
            \Dcat(F(X), Y)
            \arrow{r}[above]{\varphi_{X,Y}}
          & \Ccat(X, G(Y))
          \\
            \Dcat(F(X'), Y)
            \arrow{u}[left]{F(f)^*}
            \arrow{r}[above]{\varphi_{X',Y}}
          & \Ccat(X', G(Y))
            \arrow{u}[right]{f^*}
        \end{tikzcd}
      \end{equation}
      commutes by the naturality of~$\varphi$.
      
      The horizontal arrows in~\eqref{first diagram} are bijections, so it follows that~$F(f)^*$ is injective if~$f^*$ is injective.
      If~$f$ is an epimorphism then~$f^* \colon \Ccat(X, G(Y)) \to \Ccat(X', G(Y))$ is injective for all~$Y \in \Ob(\Dcat)$, so it then follows that~$F(f)^* \colon \Dcat(F(X), Y) \to \Dcat(F(X'), Y)$ is injective for all~$Y \in \Ob(\Dcat)$.
      This is precisely what it means for~$F(f)$ to be a epimorphism.
      This shows that~$F$ preserves epimorphisms.
      It can be shown in the same way that~$G$ preserves monomorphisms.
\end{proof}


\begin{lemma}
  If~$U \colon \Ccat \to \Dcat$ is a faithful functor then~$U$ reflects monomorphisms and epimorphisms, i.e.\ if~$U(f)$ is a monomorphism (resp.\ epimorphism) then~$f$ is a monomorphism (resp.\ epimorphism).
\end{lemma}


\begin{proof}
  Let~$f \colon X \to Y$ be a morphism such that~$U(f)$ is a monomorphism in~$\Dcat$.
  Let~$g_1, g_2 \colon Z \to X$ be two morphisms in~$\Ccat$ with~$f \circ g_1 = f \circ g_2$.
  Then
  \[
              f g_1 = f g_2
    \implies  U(f) U(g_1) = U(f) U(g_2)
    \implies  U(g_1) = U(g_2)
    \implies  g_1 = g_2
  \]
  because~$U$ is faithful.
  This shows that~$U$ reflects monomorphisms.
  That~$U$ als reflects epimorphisms can be shown in the same way.
\end{proof}


\begin{corollary}
  \label{mono epi concrete}
  Let~$\Ccat$ be a category and let~$U \colon \Ccat \to \Set$ be a faithful functor.
  Let~$f$ be a morphism in~$\Ccat$.
  If~$U(f)$ is injective then~$f$ is a monomorphism, and if~$U(f)$ is surjective then~$f$ is an epimorphism.
  \qed
\end{corollary}


It follows for the forgetful functor~$U \colon \Top \to \Set$ from \cref{adjoint mono epi} that it preserves monomorphisms and epimorphisms, because we have seen on the last exercise sheet that~$U$ is both a left adjoint and a right adjoint.
It also follows from \cref{mono epi concrete} that~$U$ reflects monomorphisms and epimorphisms, because~$U$ is faithful.
This solves the exercise.





\subsection{}

Let~$g_1, g_2 \colon Y \to Z$ be morphisms in~$\Haus$ with~$g_1 \circ f = g_2 \circ f$.
We can then consider the equivalizer
\[
    E
  = \{
      y \in Y
    \suchthat
      g_1(y) = g_2(y)
    \} \,.
\]
The set~$E$ is closed in~$Y$ because it can be written as the preimage~$E = (g_1, g_2)^{-1}(\Delta)$ for the continuous map~$(g_1, g_2) \colon Y \to Z \times Z$ and the diagonal~$\Delta \subseteq Z \times Z$, which is closed because~$Z$ is Hausdorff.
The equalizer~$E$ contains the image of~$f$ because~$g \circ f = g_2 \circ f$.
But this image is dense in~$Y$, and so it follows that
\[
            E
  =         \overline{E}
  \supseteq \overline{f(X)}
  =         Y \,.
\]
This shows that~$E = Y$ and hence that~$g_1 = g_2$.





\subsection{}

Let more generally~$(X,x_0), (Y,y_0)$ be pointed, connected topological spaces, let~$X$ be Hausdorff and let~$f \colon (X,x_0) \to (Y,y_0)$ be a morphism in~$\Conn_*$ such that the underlying continuous map~$f \colon X \to Y$ is a local homeomorphism.
Then~$f$ is a monomorphism in~$\Conn_*$:

Let~$g_1, g_2 \colon (Z,z_0) \to (X,x_0)$ be morphisms in~$\Conn_*$ with~$f \circ g_1 = f \circ g_2 \defines h$.
We then consider the equalizer
\[
            E
  \defined  \{
              z \in Z
            \suchthat
              g_1(z) = g_2(z)
            \} \,.
\]
We find as in the previous part of the exercise that the set~$E$ is closed in~$Z$, because~$X$ is Hausdorff.

The set~$E$ is also open in~$Z$:
Let~$z \in E$ and set~$x \defined g_1(z) = g_2(z)$ and~$y \defined h(z)$.
Let~$V \subseteq Y$ be an open neighbourhood of~$y$ and let~$U \subseteq X$ be an open neighbourhood of~$x$, so that~$f$ restricts to a homeomorphism~$f' \colon U \to V$.
The set~$W \defined g_1^{-1}(U) \cap g_2^{-1}(U)$ is then an open neighbourhood of~$z$ in~$Z$ for which we get the following diagram:
\[
  \begin{tikzcd}
      (W,z)
      \arrow[yshift=0.3em]{r}[above]{g_1}
      \arrow[yshift=-0.3em]{r}[below]{g_2}
      \arrow{dr}[below left]{h}
    & (X,x)
      \arrow{d}[right]{f'}
    \\
      {}
    & (V,y)
  \end{tikzcd}
\]
It follows from~$f'$ being a homeomorphism that already~$g_1 \equiv g_2$ on~$W$.
This shows that~$E$ contais an open neighbourhood (namely~$W$) around~$z$.

The set~$E$ is also nonempty because it contains the base point~$z_0$.

This shows altogether that~$E$ is a nonempty subset of~$Z$ which is both closed and open.
It follows from~$Z$ being connected that~$E = Z$ and hence that~$g_1 = g_2$.






