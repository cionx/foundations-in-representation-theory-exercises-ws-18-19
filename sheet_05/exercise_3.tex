\section{}





\subsection{}

\begin{description}
  \item[(a)~$\implies$~(d)]
    There exists a unique morphism~$Z \to Y$;
    this holds in particular for~$Y = Z$.
  \item[(d)~$\implies$~(c)]
    Both~$\id_Z$ and~$0_{\Acat(Z,Z)}$ are elements of the \dash{one}{point} set~$\Acat(Z,Z)$, hence they coincide.
  \item[(c)~$\implies$~(a)]
    It holds for every morphism~$f \colon Z \to Y$ in~$\Acat$ that
    \[
        f
      = f \circ {\id_Z}
      = f \circ 0_{\Acat(Z,Z)}
      = 0_{\Acat(X,Y)} \,.
    \]
    This shows that~$0_{\Acat(X,Y)}$ is the unique morphism~$X \to Y$ in~$\Acat$.
\end{description}
The implications (b)~$\implies$~(d)~$\implies$~(c)~$\implies$~(b) can be shown in the same way.





\subsection{}

We denote the zero object by~$Z$ (because~$0$ is already overloaded enough).
Then~$\id_Z = 0_{\Acat(Z,Z)}$ by the previous part of the exercise, and therefore
\[
    0_{X,Y}
  = 0_{Z,Y} \circ 0_{X,Z}
  = 0_{Z,Y} \circ {\id_Z} \circ 0_{X,Z}
  = 0_{Z,Y} \circ 0_{\Acat(Z,Z)} \circ 0_{X,Z}
  = 0_{\Acat(X,Y)}
\]
for all~$X, Y \in \Acat(X,Y)$.




