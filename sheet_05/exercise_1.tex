\section{}


\subsection{}

The given statement is false.
We give some counterexamples:
\begin{enumerate}
  \item
    If~$k$ is any commutative ring then
    \[
            \mat{n}{k}^\op
      \cong \mat{n}{k^\op}
      =     \mat{n}{k} \,,
    \]
    as~{\kalgs} where the first isomorphism is given by the matrix transpose~$A \mapsto A^T$.
    But the algebra~$\mat{n}{k}$ is noncommutative whenever~$n \geq 2$ and~$k \neq 0$.
  \item
    It holds for any group~$G$ that
    \[
            k[G]^\op
      \cong k[G^\op]
      \cong k[G] \,,
    \]
    as~{\kalgs} where the second isomorphism is induced by the group isomorphism
    \[
              G^\op
      \to     G \,,
      \quad   g
      \mapsto g^{-1} \,.
    \]
    But if~$G$ is noncommutative then~$k[G]$ is also not commutative.
  \item
    Consider the quiver~$Q$ with a single vertex and and two distinct arrow~$\alpha$ and~$\beta$:
    \[
      \begin{tikzcd}
        \bullet
        \arrow[loop,out=315,in=405,looseness=5]
        \arrow[loop,out=135,in=225,looseness=5]
      \end{tikzcd}
    \]
    Then~$kQ \cong k\gen{\alpha,\beta}$ is the free~{\kalg} on two generators~$\alpha$ and~$\beta$, and in particular not commutative (because~$\alpha$ and~$\beta$ don’t commute).
    But
    \[
            (kQ)^\op
      \cong k(Q^\op)
      =     kQ
    \]
    as~{\kalgs} because~$Q = Q^\op$.
  \item
    Let~$\glie$ be a Lie algebra over a field~$k$.
    Then the Lie algebra isomorphism
    \[
              \glie
      \to     \glie^\op \,,
      \quad   x
      \mapsto -x
    \]
    extends to an algebra isomorphism
    \[
            \uni(\glie)
      \to   \uni(\glie^\op)
      \cong \uni(\glie)^\op \,.
    \]
    (Here we denote by~$\uni(\glie)$ the universal enveloping algebra of~$\glie$.)
    But if~$\glie$ is not abelian (as most Lie algebras tend to be) then~$\uni(\glie)$ is noncommutative (because~$\glie$ is a Lie subalgebra of~$\uni(\glie)$).
\end{enumerate}





\subsection{}

The statement is true:
One of the~$\tensor$\nobreakdash-$\Hom$\nobreakdash-adjunctions states that the map
\[
          \Hom_B(N \tensor_A M, P)
  \to     \Hom_A(M, \Hom_B(N, P)) \,,
  \quad   f
  \mapsto [m \mapsto [n \mapsto f(m \tensor n)]
\]
is a {\welldef} isomorphism of~{\kvs}.





\subsection{}

The statement is true:
The Yoneda embedding~$Y \colon \Ccat \to \Fun(\Ccat, \Set)$ is fully faithful and thus reflects isomorphisms.
It therefore follows for any two ojects~$X, X' \in \Ob(\Ccat)$ that the map
\[
        \{\text{isomorphisms~$X \to X'$}\}
  \to   \{\text{natural isomorphisms~$h^{X'} \to h^X$}\} \,,
  \quad f
  \mapsto f^*
\]
is a {\welldef} bijection.
It therefore follows from~$h^X \cong h^{X'}$ that also~$X \cong X'$





\subsection{}

The statement is true:
Suppose more generally that~$\Ccat$ is a category which admits an intinal object~$I$.
Then the functor~$F \colon \Ccat \to \Set$ given by~$F(A) = \{\ast\}$ is represented by~$I$:
There exist for every object~$X \in \Ob(\Ccat)$ a (unique) bijection
\[
          \eta_X
  \colon  h^I(X)
  \to     F(X) \,.
\]
The square
\[
  \begin{tikzcd}
      h^I(X)
      \arrow{r}[above]{f_*}
      \arrow{d}[left]{\eta_X}
    & h^I(I')
      \arrow{d}[right]{\eta_{X'}}
    \\
      F(X)
      \arrow{r}[above]{F(f)}
    & F(X')
  \end{tikzcd}
\]
commutes for every morphism~$f \colon X \to X'$ in~$\Ccat$ because the set~$F(X')$ is a singleton.
This shows that~$\eta \colon h^I \to F$ is a natural isomorphism, hence that~$I$ represents the functor~$F$.

In our case~$\Ccat = \kAlg$ the initial object, and hence representing object of~$F$, is given by the~{\kalg}~$k$.





\subsection{}

The statement is true:
Let~$(X^\lambda)_{\lambda \in \Lambda}$ be a family of representations of~$Q$ over~$k$.
We can then define a new representation~$\prod_{\lambda \in \Lambda} X^\lambda$ of~$Q$ over~$k$ by setting
\[
            \left( \prod_{\lambda \in \Lambda} X^\lambda \right)_i
  \defined  \prod_{\lambda \in \Lambda} X^\lambda_i
\]
for every~$i \in Q_0$, and
\[
            \left( \prod_{\lambda \in \Lambda} X^\lambda \right)_\alpha
  \defined  \prod_{\lambda \in \Lambda} X^\lambda_\alpha
  \colon    \prod_{\lambda \in \Lambda} X^\lambda_{s(\alpha)}
  \to       \prod_{\lambda \in \Lambda} X^\lambda_{t(\alpha)}
\]
for every~$\alpha \in Q_1$.

Let~$\lambda \in \Lambda$, and for every~$i \in Q_0$ let~$p^\lambda_i \colon \prod_{\mu \in \Lambda} X^\mu_i \to X^\lambda_i$ be the canonical projection.
Then the family~$p^\lambda \defined (p^\lambda_i)_{i \in Q_0}$ is a morphism of representations~$p^\lambda \colon \prod_{\mu \in \Lambda} X^\mu \to X^\lambda$ because the square
\[
  \begin{tikzcd}[column sep = huge]
      \prod_{\lambda \in \Lambda} X^\lambda_{s(\alpha)}
      \arrow{r}[above]{\prod_{\lambda \in \Lambda} X^\lambda_\alpha}
      \arrow{d}[left]{p^\lambda_{s(\alpha)}}
    & \prod_{\lambda \in \Lambda} X^\lambda_{t(\alpha)}
      \arrow{d}[right]{p^\lambda_{t(\alpha)}}
    \\
      X^\lambda_{s(\alpha)}
      \arrow{r}[above]{X^\lambda_\alpha}
    & X^\lambda_{t(\alpha)}
  \end{tikzcd}
\]
commutes for every arrow~$\alpha \in Q_1$.

Suppose now that~$Y$ is a representation of~$Q$ over~$k$ and that for every~$\lambda \in \Lambda$ we are given a morphism of representations~$f^\lambda \colon Y \to X^\lambda$.
There exist for every~$i \in Q_0$ a unique~{\klin} map~$g_i \colon Y_i \to \prod_{\lambda \in \Lambda} X^\lambda_i$ with~$p^\lambda_i \circ g_i = f^\lambda_i$ for every~$\lambda \in \Lambda$.
The family~$g = (g_i)_{i \in Q_0}$ is a morphism~$g \colon Y \to \prod_{\lambda \in \Lambda} X^\lambda$ because the square
\begin{equation}
  \label{g is a morphism}
  \begin{tikzcd}[column sep = huge]
      \prod_{\lambda \in \Lambda} X^\lambda_{s(\alpha)}
      \arrow{r}[above]{\prod_{\lambda \in \Lambda} X^\lambda_\alpha}
    & \prod_{\lambda \in \Lambda} X^\lambda_{t(\alpha)}
    \\
      Y_{s(\alpha)}
      \arrow{u}[left]{g_{s(\alpha)}}
      \arrow{r}[above]{Y_\alpha}
    & Y_{t(\alpha)}
      \arrow{u}[right]{g_{t(\alpha)}}
  \end{tikzcd}
\end{equation}
commutes for every arrow~$\alpha \in Q_1$.
Indeed, we have for every~$\mu \in \Lambda$ the following diagram:
\[
  \begin{tikzcd}[column sep = huge]
      X^\mu_{s(\alpha)}
      \arrow{r}[above]{X^\mu_\alpha}
    & X^\mu_{t(\alpha)}
    \\
      \prod_{\lambda \in \Lambda} X^\lambda_{s(\alpha)}
      \arrow{u}[left]{p^\mu_{s(\alpha)}}
      \arrow{r}[above]{\prod_{\lambda \in \Lambda} X^\lambda_\alpha}
    & \prod_{\lambda \in \Lambda} X^\lambda_{t(\alpha)}
      \arrow{u}[right]{p^\mu_{t(\alpha)}}
    \\
      Y_{s(\alpha)}
      \arrow[bend left = 90]{uu}[left]{g^\mu_{s(\alpha)}}
      \arrow{u}[left]{g_{s(\alpha)}}
      \arrow{r}[above]{Y_\alpha}
    & Y_{t(\alpha)}
      \arrow[bend right = 90]{uu}[right]{g^\mu_{t(\alpha)}}
      \arrow{u}[right]{g_{t(\alpha)}}
  \end{tikzcd}
\]
The triangles on the left and right commute by construction of~$g$, the upper square commutes by construction of~$p^\mu$, and the outer square commutes because~$g^\mu \colon Y \to X^\mu$ is a morphism of representations.
It follows that
\[
    p^\mu_{t(\alpha)} \circ \prod_{\lambda \in \Lambda} X^\lambda_\alpha \circ g_{s(\alpha)}
  = p^\mu_{t(\alpha)} \circ g_{t(\alpha)} \circ Y_\alpha \,.
\]
This holds for every~$\mu \in \Lambda$, and so it follows that
\[
    \prod_{\lambda \in \Lambda} X^\lambda_\alpha \circ g_{s(\alpha)}
  = g_{t(\alpha)} \circ Y_\alpha \,.
\]
This shows that the square~\eqref{g is a morphism} commutes, and hence that~$g$ is a morphism of representations.

It holds that~$p^\lambda \circ g = f^\lambda$ because
\[
    (p^\lambda \circ g)_i
  = p^\lambda_i \circ g_i
  = f^\lambda_i
\]
at every~$i \in Q_0$ by construction of~$g$.
That~$g$ is the unique morphism of representations~$Y \to \prod_{\lambda \in \Lambda} X^\lambda$ with this property follows from the uniqueness of the components~$g_i$.

This shows altogether that the constructed representation~$\prod_{\lambda \in \Lambda} X^\lambda$ is together with the projections~$p^\mu \colon \prod_{\lambda \in \Lambda} X^\lambda \to X^\mu$ a product of the family~$(X^\lambda)_{\lambda \in \Lambda}$.


\begin{remark}
  One can also somewhat avoid this explicit calculations:
  \begin{enumerate}
    \item
      By using that~$\Rep{k}{Q} \cong \Fun(\Path(Q), k)$ as categories one can argue that the functor category~$\Fun(\Path(Q), k)$ inherits all nice properties from the nice category~$\Modl{k}$.
      But we haven’t shows in the lecture that a functor category~$\Fun(\Ccat, \Dcat)$ inherits products from~$\Dcat$, and checking this by hand amounts to the above calculations.
    \item
      One can argue that~$\Rep{k}{Q} \cong \Modl{kQ}$, but~$kQ$ is for a general quiver~$Q$ (no finiteness conditions on~$Q$ are given on the exercise sheet) not a unital~{\kalg}, which may make some people uncomfortable.
  \end{enumerate}
\end{remark}





\subsection{}

The statement is false:
If the category~$\Field$ would have a terminal object~$K$ then there would exist (unique) field homomorphisms~$\Finite_2 \to K$ and~$\Finite_3 \to K$.
But then~$K$ would need to have both characteristic~$2$ and characteristic~$3$, which cannot be.




